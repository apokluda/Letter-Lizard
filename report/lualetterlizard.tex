\subsection{LuaLetterLizard: Lua Letter Lizard Implementation}
\label{luaimpl}

Lua is a light weight programming language designed as a scripting language. Lua is generally described as a ``multi-paradigm'' programming language because it provides a small set of extensible features that can be extended to fit various problems rather than providing definite features for a specific paradigms. Lua has a short learning curve and is easy to pick up. Lua has a number of features which endorse its reputation. Lua, for example, does not support inheritance but allows it to be implemented by using metatables.
A table is a fundamental data structure in Lua that can be used to represent everything. Tables are in fact the only data structure in Lua and can be used to implement other data structures such as arrays, sets, lists, records and other data structures. Lua tables can be used to implement features such as namespaces and classes. Lua shares a lot of similarities with JavaScript. For instance, tables in Lua are quite similar to JavaScript objects--the only difference being that in JavaScript objects can be indexed with only string or integer values but Lua tables can be indexed with any value of the language (except nil). 
Lua is dynamically typed and is light enough to fit on any operating system. The Lua interpreter is extremely lightweight, it is only 180 kB when compiled. Hence, Lua is very fast as compared to Python and JavaScript. It is a minimal but powerful language. By including only a minimal set of data types, Lua attempts to strike a balance between power and size \cite{about_lua}.
Lua is often used in game development because of its speed, portability, embeddable nature and simple but powerful design. In order to enable embedding in other languages, Lua provides a well documented API to extend programs written in other languages. 

A number of important language features in Lua include first-class functions, dynamic module loading, automatic coercion and closures. Although Lua does not support the object-oriented concepts of classes and methods, they can be simulated using tables and first-class functions. By placing functions and related data into a table, objects are formed. Inheritance can be implemented using metatables. It is similar to JavaScript in this way as there is no explicit concept of a class in Lua, rather prototypes are used similar to JavaScript and new objects are created by a factory method, i.e, by cloning existing objects. Lua provides ``syntactic sugar'' to facilitating object-orientated programming techniques. In order to declare member functions inside a table, the programmer can use \texttt{table:func(args)}. The colon operator adds a hidden ``self'' parameter to function calls.

\subsubsection{Lua Game Framework}
We used a framework called L\"OVE for developing the Lua-based implementation of Letter Lizard. L\"OVE is a free, open source framework for building 2D games in Lua. The L\"OVE framework has cross platform adaptability and was installed from the L\"OVE homepage~\cite{about_love}. To run a game, L\"OVE can load a game in two ways, i.e, from a folder that contains a \texttt{main.lua} file or from a \texttt{.love} file that has a \texttt{main.lua} in the root directory. L\"OVE utilizes callback functions to perform various tasks. L\"OVE provides placeholders for callback functions in order to structure the game logic. For instance, the \texttt{love.load} function is called only once when the game is started and is usually used to to load resources, initialize variables and set specific settings. The \texttt{love.draw} function is where all the drawing happens and if any of the \texttt{love.graphics.draw} objects are called outside of this function, they will not have any effect. Figure~\ref{luascreenshots1} shows some screenshots from the Lua version of Letter Lizard.

\begin{figure}
    \centering
    \begin{subfigure}{0.49\textwidth}
        \includegraphics[width=\textwidth]{../screenshots/luasplash.png}
        \caption{Splash Screen}
        \label{luasplash}
    \end{subfigure}
    \begin{subfigure}{0.49\textwidth}
        \includegraphics[width=\textwidth]{../screenshots/luabegin.png}
        \caption{Game Screen}
        \label{luabegin}
    \end{subfigure}
    \begin{subfigure}{0.49\textwidth}
        \includegraphics[width=\textwidth]{../screenshots/luagameplay.png}
        \caption{Gameplay Mode}
        \label{luagameplay}
    \end{subfigure}
    \begin{subfigure}{0.49\textwidth}
        \includegraphics[width=\textwidth]{../screenshots/luagameover.png}
        \caption{Game Over}
        \label{luagameover}
    \end{subfigure}
    \caption[Four screenshots from the Letter Lizard Lua implementation]
    {Four screenshots from LuaLetterLizard showing (a) the splash screen, (b) the     
    main game screen, (c) the ``Good Job!'' message that is displayed after the player
    finds a word, and (d) the ''Time's Up!'' message that is displayed when
    the timer runs out.}
    \label{luascreenshots1}
\end{figure}

The game structure consists of five different Lua modules. These are \texttt{main.lua}, \texttt{conf.lua}, \texttt{games.lua},  \texttt{helper\_functions.lua}, \texttt{gamestate.lua} and \texttt{config.lua}. 
The \texttt{main.lua} module is the driver of the game and utilizes all the other modules. \texttt{conf.lua} contains game configurations and is run exactly once before \texttt{main.lua} by L\"OVE. \texttt{games.lua} contains 300 games pre-generated using game generator program. \texttt{gamestate.lua} was taken from  a small Lua utility library and enables us to maintain state information within the game and allow us to switch between states. Since Lua is a very low-level language, we had to code a lot of functionality on our own. These free standing functions are contained in the \texttt{helper\_functions.lua} module. \texttt{config.lua} contains additional game configurations which are used by \texttt{main.lua} to drive the game. In order to execute the game we need to run the L\"OVE executable and give the directory containing the \texttt{main.lua} file as an argument. LuaLetterLizard can be launched by running the command \texttt{love LuaLetterLizard} in our \texttt{uwaterloo-cs798scripting / group4} Github repository directory.
	
	\subsubsection{Game Logic and Implementation}
	
	The \texttt{main.lua} file contains the main game logic structured within callback functions. In this file, we declare two Lua table structures, menu and game. We implement metatables a powerful feature provided by Lua, to enable the concept of state within the game. Within the menu table we declare functions which are associated with the menu and are called only when menu is the current state of the game. Within the game structure, we declare functions which are called only when the we enter the gameplay mode.

The \texttt{menu} table contains functions that are called when we start the game and the splash screen is loaded, as shown below. The function \texttt{menu:init()} is called when the game is first loaded into memory and the splash screen is displayed. In order to actually draw the splash screen, the function \texttt{menu:draw()} is called. Within the \texttt{love.keypressed()} function we check if the player has hit the space bar, and if so, the state of the game is switched to ``game'' and the associated callback functions are then called. All of the game play logic is structured within the \texttt{game} table. L\"OVE is different from Pygame in that there is no infinite loop which keeps running. Rather, callbacks are called as and when required. However, there is still an update function, \texttt{love.update()}, that is called to render each frame of the game. It is passed a parameter \texttt{dt} which is the time passed since the last time update was called.

%\noindent\begin{minipage}[t]{\textwidth}
%\centering
\begin{lstlisting}[language={[5.2]Lua},caption=LuaLetterLizard \texttt{menu:init()} function,label=splash]
local menu = {}
function menu:init()
    splash = love.graphics.newImage("splash.png")
end

function menu:draw()
    love.graphics.setColor(255,255,255,255)
    love.graphics.draw(splash, 0 ,0)
end
\end{lstlisting}
%\end{minipage}

Once we enter into the game mode, the corresponding callback functions defined in the \texttt{game} table are called. For instance, in the following code snippet, the function \texttt{game:draw()} is called to draw objects onto the screen while we are currently in the gameplay state i.e., the default \texttt{love.draw()} function is overridden. Hence, callbacks defined when in a particular state are called as and when required.

%\noindent\begin{minipage}[t]{\textwidth}
%\centering
\begin{lstlisting}[language={[5.2]Lua},caption=LuaLetterLizard \texttt{game:draw()} callback method, label=game]
function game:draw()
    love.graphics.setColor(black)
    love.graphics.line(700,0, 700, 500)
    for i, letter in ipairs(letters_guessed) do
        x = letters_guessed_left + i * square_width + i * spacing
        y = letters_guessed_top
        love.graphics.rectangle("line", x, y, square_width, square_width)
        love.graphics.setFont(number_fnt_40)
        love.graphics.print(letter, x + square_width/4, y + square_width/5)
    end
    ...
end
\end{lstlisting}
%\end{minipage}

In order to maintain the game state, we utilized an external Lua library called ``hump.'' This is a lightweight library with contains helper code for embedding a set of different functionality within our code. Since Lua is very low level, we had to code a lot of functionality ourselves, and using hump minimized a lot of extra work for us considering the time constraints. This was another advantage of using the L\"OVE framework, as L\"OVE has a very good online community and many helpful libraries have been implemented by other Lua game developers and can be used under the open source licenses.
Another important part of the game logic is checking for events. For this, we utilized the callback functions defined by L\"OVE, \texttt{love.keypressed()} and \texttt{love.mousepressed()}. Within the former callback we check for events that fire when a keyboard key is pressed by the player and define logic for the corresponding action that needs to be taken whereas in the later we check for events when the player click a particular button which fire the corresponding actions. In the following code snippet, we check if the space key has been pressed by the player while we are in state ``menu'' and if so, we switch the game state to ``game'' to begin playing the game.
	
%\noindent\begin{minipage}[t]{\textwidth}
%\centering
\begin{lstlisting}[language={[5.2]Lua},caption=Handling the keypressed event in LuaLetterLizard,label=keypressed]
function menu:keypressed(key)
    if key == ' ' then
        Gamestate.switch(game)
    end
end
\end{lstlisting}
%\end{minipage}
