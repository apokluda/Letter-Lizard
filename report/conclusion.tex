\section{Conclusion}
\label{conclusion}

\lettrine[nindent=0em,lines=3]{W}{e} have gained a deeper understanding and
appreciation for scripting languages by implementing a non-trivial program,
the Letter Lizard game, in three different scripting languages and comparing the
implementations. We found that it was fairly easy to implement the game in
all three languages, and that the effort required was less that would have been
required to implement the game in a static language such as C, C++ or Java.
Of the three implementations, we found that JavaScript required the least
amount of effort due to the high-level functionality and event-driven nature 
of the client-side scripting environment provided by Web browsers. Of all the features
that we used, we found first-class functions and closures to be the most useful. 
First-class functions and closures made writing callback functions for event-driven
programming simple and easy, and led to a clean, modular design. We avoided some of
the negative features of JavaScript, such as automatic semicolon insertion, by
always finishing statements with an explicit semicolon. Although JavaScript objects
are not as powerful as Lua tables, we did not encounter any limitations in our
implementation. Similarly, although Python's ``broken lexical scoping'' may cause
problems for some programs, we did not encounter any difficulty with our implementation.
Overall, we gained a deeper understanding of, and appreciation for, scripting languages
and their ability to enable rapid application development and we will likely use
scripting languages more often in the future.