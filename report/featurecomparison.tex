\section{Scripting Language Feature Comparison}
\label{comparison}

% Compare language features. Show examples for one or more language implementations
% whenever possible. The structure of this section will be a feature-by-feature comparison
% first, backed up by examples (or hypothetical examples) where possible.

% comparing the most major differences that we noticed using examples from our three implementations (when possible)
% don't have space to go into all the details
% First difference noted already: environments/frameworks led to different design

% NOT COVERED:
% Many differences: Types, Variables, Values (including Numeric and Boolean types, Strings)
% Special Types such as (nil, null and undefined)

% COVERED:
% Data Structures: arrays, lists, tuples, dictionaries (Python); Objects (JavaScript); Tables (Lua)

% NOT COVERED:
% Expressions and Statements, Flow Control

% COVERED: Variable Scope

% COVERED: Functions, closures, coroutines, generators, iterators

% NOT COVERED: Exception Handling, Modules and Packages

\lettrine[nindent=0em,lines=3]{N}{ow} that we have introduced the three implementations of
our Letter Lizard game, we will will compare and contrast the differences between
each language, noting their strengths and weaknesses, using example code from our
implementations where possible. When coding the Letter Lizard game in each language,
we tried to follow the same structure as much as possible, however, as was noted in section~\ref{lljs}
the event-driven environment of client-side JavaScript naturally led to a different
design for that implementation. 

\subsection{Lexical Structure}
% LEXICAL STRUCTURE
% STATEMENTS
% - Python: intentation sensitive; JavaScript: automatic semicolon insertion; 
%   Lua neither
% LUA: CHUNKS
% - Chunks: Lua specific thing
% COMMENTS

% AUTHOR: ALEX OR VOLUNTEER?

\subsubsection{Data Structures}
% - Arrays, Lists, Tuples, Dictionaries
%   - only Python has all of these, Lua and JavaScript have tables/objects with 
%     special properties (covered under Objects)
%   - list comprehensions

% AUTHOR (INTRO): ALEX OR VOLUNTEER?

\paragraph{Python: Arrays, Lists, Tuples and Dictionaries}

% AUTHOR: MIKE

\paragraph{JavaScript: Objects}

% AUTHOR: ALEX

\paragraph{Lua: Tables}

% AUTHOR: AFIYA

\subsection{Variable Scope}
\label{varscope}
% VARIABLE SCOPE
% - local variables vs global variables
% - block scope vs function scope
% - Lua do blocks

% AUTHOR (PARAGRAPH ON JAVASCRIPT): ALEX
% AUTHOR (PARAGRAPH ON PYTHON AND LUA TOGETHER): MIKE

\subsection{Functions}
% FUNCTIONS
% - how are arguments passed? (Python: must number of arguments match? Doesn't have to 
% in JavaScript and Lua)
% - Python's name-based arguments vs passing objects in JavaScript / Lua
% - scope
% - return value
% - functions as values

% AUTHOR (INTRO): ALEX OR VOLUNTEER?

\subsubsection{Closures}
\label{closures}
% - closures, coroutines

% AUTHOR (PARAGRAPH ON JAVASCRIPT): ALEX
% AUTHOR (PARAGRAPH ON LUA): AFIYA

\subsubsection{Generators and Iterators}
% GENERATORS, ITERATORS

% AUTHOR: MIKE (OR REMOVE THIS SECTION)

\subsection{Object-Oriented Programming}
\label{oop}
% OBJECT ORIENTED PROGRAMMING
% - class-based inheritance vs prototype based inheritance
% - private data
% - JavaScript: no distinction between func and method: caling convention (same with Lua)

% AUTHOR (PARAGRAPH ON JAVASCRIPT): ALEX
% AUTHOR (PARAGRAPH ON PYTHON): MIKE
% AUTHOR (PARAGRAPH ON LUA): AFIYA