\section{Scripting Language Feature Comparison}
\label{comparison}

\lettrine[nindent=0em,lines=3]{N}{ow} that we have introduced the three implementations of
our Letter Lizard game, we will will compare and contrast the differences between
each language, noting their strengths and weaknesses, using example code from our
implementations where possible. When coding the Letter Lizard game in each language,
we tried to follow the same structure as much as possible, however, as was noted in section~\ref{lljs}
the event-driven environment of client-side JavaScript naturally led to a different
design for that implementation. Due to space constraints, we will skip over some of
the differences that we noticed between the types, variables and values (including 
the numeric, boolean and string types and the null, undefined and nil special values)
offered by each language.
We will, however, discuss the differences between the data structures offered by
Python (i.e. lists, tuples and dictionaries), JavaScript (i.e. objects and arrays)
and Lua (i.e. tables and arrays). We noted differences between the syntax and semantics
of expressions and statements in each language, as well as the flow
control constructs offered, which we will omit; however, we will discuss
variable scope, functions, and closures. The first major difference that we
noticed between the languages was non-functional: each employed a different lexical
structure, which we discuss next.

\subsection{Lexical Structure}
\label{lexstruct}

Since we implemented the game in three different languages we were able to examine the difference in lexical structure quite minutely. With regard to programming languages, the syntax of the language defines the set of symbols that are considered to be a correctly structured document in that particular language. Python is inherently designed to be a highly readable language and uses words of the English language instead of punctuation marks. As compared to other languages, Python uses white space indention rather than curly braces or keywords to delimit blocks. This is a strength of the language as good indention is enforced by the language itself and makes for highly readable, clean code. However, on the downside, tabs and spaces can be easily mixed up leading bugs in the code. 

JavaScript uses blocks delimited by curly braces which means that code can be ``minimized'' for the Web. However, automatic semicolon insertion can lead to errors and is generally not considered very good by JavaScript programmers. This is one of the controversial syntactic features of JavaScript. For instance, consider the following code snippet:
\begin{lstlisting}[language={JavaScript},caption=An example where automatic semicolon insertion in JavaScript may lead to unexpected results]
	a = b + c
   (d + e).print()
\end{lstlisting}
In this example, the code is not transformed by automatic semicolon insertion, because the parenthesized expression that begins the second line can be interpreted as an argument list for a function call:
\begin{lstlisting}[language={JavaScript},caption=The interpretation of the previous example]
	a = b + c(d + e).print()
\end{lstlisting}

In Lua the statements and blocks are delimited by newlines and keywords to begin and end constructs which helps to minimize errors, but it can lead to extremely messy code unless indention conventions are strictly followed.

% AUTHOR: AFIYA

\subsubsection{Data Structures}

% AUTHOR (INTRO): AFIYA
A data structure is a particular way of storing and organizing data in a computer program so that it can be used efficiently. There are different kinds of data structures available which are suited for performing different tasks. In our implementations of Letter Lizard, we extensively used the different data structures provided by each language for holding, structuring and manipulating our data efficiently. All three of our implementations used data structures for holding the scrambled puzzle letters, letters guessed correctly, letters to be displayed on the screen and so on. We will now describe the data structures in all three of our implementations.

\paragraph{Python: Arrays, Lists, Tuples and Dictionaries}
% - Arrays, Lists, Tuples, Dictionaries
%   - only Python has all of these, Lua and JavaScript have tables/objects with 
%     special properties (covered under Objects)
%   - list comprehensions

% AUTHOR: MIKE

\paragraph{JavaScript: Objects}

JavaScript has one fundamental datatype: Object. Everything that is not either a
primitive type, null or undefined is an object. (The primitive types in JavaScript are
number, string and boolean. Although they are not objects, they behave like immutable
objects). Objects are composite values that aggregate multiple values, which may be
primitive types or other objects, and allow you to store and retrieve those values
by name. Objects are similar to Python's dictionary and Lua's table data structure, but
unlike Python's dictionary, Objects are a core feature of JavaScript. All variables are
contained within objects and, as we will see later in section~\ref{functions}, even
the scope of a variable is implemented in terms of a chain (or linked list) of objects.
Lua's tables are very similar to JavaScript objects and play a similar role in the
implementation of the language, but they are even more powerful.

JavaScript objects are dynamic--rather than having a static type, properties can be added
and removed at run-time and support ``duck typing.'' LetterLizardJS makes use of many
user-defined objects. The following code defines an object \texttt{letterpoints} that
maps letters (property names) to the number of points that the letter is worth
(property values) when included in a word found by the player:

\begin{lstlisting}[caption=A user-defined object in JavaScript]
var letterpoints = {
	'A': 3,
	'B': 8,
	'C': 6,
	...
	'Z': 10
};
\end{lstlisting}

Another example of a user-defined object in LetterLizardJS is shown below. This code
snippet is from the Game constructor function. It adds a property called 
\texttt{words} to the Game object being constructed and assigns an empty object
to it on line 2. Then it iterates through a randomly chosen ``game'' (i.e. a
randomly chosen set of letters and corresponding words to be found) and creates
a Word object to represent each word to be found. Word objects draw themselves to the
screen either as text or a placeholder when the word is yet to be found. The Word
objects are assigned to the words object by dynamically creating properties mapping
the text of each word to its corresponding Word object.

\begin{lstlisting}[caption=A user-defined object demonstrating dynamic properties]
	var game = games[config.difficulty][i];
	this.words = {};
	for (var i = 0; i < game.words.length; ++i) {
		var word = game.words[i];
		this.words[word] = new Word(word);
	}
\end{lstlisting}

JavaScript also provides special support for using Objects as Arrays. When an Object 
is created using the array literal syntax [] or Array constructor, \texttt{Array()}
and property names are integer values, the language provides a \texttt{length} property
on the object that is automatically updated to reflect the number of elements in the array.

\paragraph{Lua: Tables}

% AUTHOR: AFIYA
Lua offers a single, fundamental data structure: \texttt{Table}. Tables are the only data structure in Lua. All the other structures that the language offers can be represented as Tables efficiently. In traditional languages like C and C++, most of the structures are represented with arrays and lists. However, in Lua, tables are more powerful than either. The algorithms used for implementing these structures in traditional languages are simplified with the use of tables. Tables offer direct access to any type. Tables are fundamentally associative arrays, i.e., they are key-value pairs in which the keys can be any legitimate value in Lua. Tables are quite similar to the Object type of JavaScript, the difference being that with JavaScript objects the keys can be only either strings or integers, whereas in Lua tables the keys can be any value in Lua except nil. We now describe the different structures in Lua that we defined using tables.
Tables are very easy to create. An empty table can be assigned to any variable by specifying empty curly braces which denotes an empty table constructor. For instance, throughout our code, we have declared tables easily and efficiently as shown in the following code.

\begin{lstlisting}[language={[5.2]Lua},caption=Declaring tables in Letter lizard]
	  games_letters = {}
	  letters_guessed = {}
	  solutions = {}
	  words_guessed_correct = {}
\end{lstlisting}

Arrays in Lua can be implemented efficiently using Lua tables by simply indexing the tables with integers. Hence, arrays do not have a finite size, instead they can grow in size as needed. The following code snippet is an example of an array implementation from Lua Letter Lizard. By convention, array indexing starts from 1 in Lua.

\begin{lstlisting}[language={[5.2]Lua},caption=An array structure, \texttt{games.easy}, that holds pre-generated game puzzles and is easily accessed by indexing into the table]
     games_words = games.easy[i].words
\end{lstlisting}

Metatables are a very powerful feature offered by Lua. Metatables allow us to change the behaviour of a table. For instance, using metatables we can define how Lua computes the expression \texttt{a+b} where \texttt{a} and \texttt{b} are tables and establish prototype chains like we have in JavaScript. 

% In our code we have used metatables to declare action to be taken when we are in a particular state. In the following code, we declare two tables \texttt{menu} and \texttt{game} corresponding to the states ``menu'' and ``game.'' We define callback functions within these state tables which override the default callbacks provided by L\"OVE and are called when the game is in a particular state. Functions are declared within the two tables as easily as \texttt{function menu:init()} or \texttt{function game:keypressed()}.
%
%\begin{lstlisting}[language={[5.2]Lua},caption= Declaring and initializing metatables in Lua]
%local menu = {}
%function menu:init()
%    splash = love.graphics.newImage("splash.png")
%end
%
%function menu:draw()
%    love.graphics.setColor(255,255,255,255)
%    love.graphics.draw(splash, 0 ,0)
%end
%
%function menu:keypressed(key)
%    if key == ' ' then
%        Gamestate.switch(game)
%    end
%end
%\end{lstlisting}


\subsection{Variable Scope}
\label{varscope}

Overall, we discovered that Lua and JavaScript have many more similarities with
each other than they do with Python; however, variable scope is one area
where this differs. Both Python and JavaScript implement function scoping. The only
means to create a scope for a variable in Python is a function, class or module and
only a function in JavaScript. The official Python documentation states: 
``If a name binding operation occurs anywhere within a code block, all uses of the 
name within the block are treated as references to the current block~\cite{pyscope}.'' 
This subtle point can lead to surprising results, especially for programmers not
intimately familiar with the scoping rules in these languages. Consider the
following example:

\begin{lstlisting}[language=Python,caption=A demonstration of function scope in Python]
a = 42
def f():
	print(a)
	if (True):
		a = 43
f()
\end{lstlisting}

Most programmers familiar with the block scoping rules of C, C++, and Java would expect
this code to print 42; however, it results in an \texttt{UnboundLocalError} because
the assignment on line 5 creates a new local variable called ``a'' that shadows the
variable with the same name at the global scope and is visible throughout the
entire function, but when line 3 is executed, this variable has not yet been assigned
a value. Also, because Python lacks an explicit variable declaration statement,
it is not clear whether the assignment on line 5 was intended to create a new
local variable or change the value of the global variable. In fact, prior to Python 3,
there is no way for code in an inner scope to assign a value to an enclosing scope
that is not the global scope. (Python 3 fixes this with the introduction of the 
\texttt{nonlocal} statement).

JavaScript uses function scoping for variables, similar to Python, which means that all 
variable declared in a function are visible throughout the entire body of the function.
Unlike Python, though, JavaScript has an explicit variable declaration statement and
this feature is known as ``hoisting:'' JavaScript code behaves as if all 
variable declarations in a function are hoisted to the top of the function. In both
languages, this feature can easily cause bugs that are hard to find when local variables
shadow variables in an outer scope, and for this reason it is considered a negative 
feature~\cite{goodparts}.

On the other hand, Lua implements block scope similar to C, C++ and Java.
The body of a control structure, body of a function, or chunk (a segment of
code that is treated as a unit) all introduce new scopes. Additionally,
new scopes can be created explicitly with the keywords \texttt{do} ... \texttt{end}.
The scope of a declared variable begins after the declaration and goes until the
end of the block. When written in Lua, as shown below, the previous example
will print 42. 

\begin{lstlisting}[language={[5.2]Lua},caption=A demonstration of block scope in Lua]
a = 42
function f()
	print(a)
    if (true) then
        local a = 43
    end
end
f()
\end{lstlisting}

\subsection{Functions}
\label{functions}
% AUTHOR (INTRO): ALEX

All three languages support first-class functions: functions can be stored
in variables, passed as arguments to other functions, and be returned
from functions as results. First class functions give these languages
great flexibility: functions can be redefined to change their behaviour,
or even erased to create a secure environment for untrusted code. Perhaps
most importantly, functions can be nested and have access to the scope where
they were defined. This means that functions in these languages are \emph{closures}
and this enables some important and powerful techniques that will be discussed next.

\subsubsection{Closures}
\label{closures}
% AUTHORS: AFIYA AND ALEX

%Closures are a powerful tool offered by any programming language. Closures are basically functions written within functions such that they have access to the local variables from the enclosing function. All three languages support closures. These act as a valuable tool in many contexts. In our implementation, closures were very useful for implementing callback functions in creating the GUI buttons. Each button has a callback function to be called when we create the button. For instance in the following code snippet, we utilize an external GUI library for Love called \texttt{LoveFrames} to create the 'New Game' button. LoveFrames uses closure function to 'create' the button, set its positions and define the button name.
%\begin{lstlisting}[language={[5.2]Lua},caption=Declaring and initializing metatables in Lua]
%newgame_button = loveframes.Create("button")
%    newgame_button:SetSize(button_width, button_height)
%    newgame_button:SetPos(100, 400)
%    newgame_button:SetText("NEW GAME")
%\end{lstlisting}

In addition to first-class functions, all three languages also support closures.
Every function also contain a reference to the scope chain that was
in effect when the function was defined, which is used to resolve variable
names to values when the function is executed. A function, together with a
reference to its scope chain is known as a \emph{closure}.
All three implementations make use of first-class functions and closures for 
implementing callbacks, especially the JavaScript version due to the 
event-driven nature of the client-side scripting environment in the Web browser.

The following example from LuaLetterLizard demonstrates the use of a closure
for registering a set of callbacks to be called by the LOVE game library when
certain events occur. The variable \texttt{registry} is defined in \texttt{registerEvents}
but also accessed in an anonymous inner function, which is made possible by the closure.

\begin{lstlisting}[language={[5.2]Lua},caption=A closure in Lua]
function GS.registerEvents(callbacks)
	local registry = {}
	callbacks = callbacks or all_callbacks
	for _, f in ipairs(callbacks) do
		registry[f] = love[f] or __NULL__
		love[f] = function(...)
			registry[f](...)
			return GS[f](...)
		end
	end
end
\end{lstlisting}

The next example is from LetterLizardJS. A callback function is assigned to the
timer object's \texttt{ontimeup} property that is called when the time remaining
reaches zero. This code snippet is contained within the Game constructor and the
inner function needs access to the Game objects, so we first store the value of this
(which refers to the Game object) in a variable called \texttt{that}, which will be
contained within the closure and accessible to the inner function.

\begin{lstlisting}[caption=A closure in JavaScript]
	var that = this;
	this.timer.ontimeup = function() {
		timeup = true;
		for (var word in that.words) {
			that.words[word].show(true);
		}
		that.nextRoundOrEnd();
	};
\end{lstlisting}

\subsection{Object-Oriented Programming}
\label{oop}

% AUTHOR (PARAGRAPH ON JAVASCRIPT): ALEX

All three languages take a different approach to object-oriented programming.
Python supports a traditional class-base inheritance model, while JavaScript
supports a protoype-based inheritance model. 

\paragraph{Python}
% AUTHOR (PARAGRAPH ON PYTHON): MIKE

\paragraph{JavaScript}
JavaScript supports object-oriented programming, but not in the classical
sense. Rather than providing class-based inheritance, JavaScript provides
\emph{prototype-based inheritance}. In addition to having their own set of 
properties (known as ``own properties'') every object is associated with and
inherits properties from another object called its \emph{prototype}
indicated by its prototype property (occasionally the value of the prototype
property will be null, but this is rare). 

Many programmers who are new to JavaScript who come from a
C, C++, or Java background find JavaScript's prototype-based inheritance
confusing at first; however, it is really quite simple and works nicely
JavaScript's dynamic nature. If a requested property is not an own property
of an object, then its prototype, its prototype's prototype, etc. are searched
recursively until the property is found. Most classical object-oriented features
including static class-based objects, structs and inheritance hierarchies can be
easily simulated in JavaScript. 

Prototypal inheritance is a key feature of JavaScript and LetterLizardJS makes
extensive use of objects and protoypes. The following shows the definition
of the Tile constructor function and its prototype object. The constructor function
is used to create a \emph{class} of objects that have their own properties to
hold their state, but share behaviour inherited from the prototype object:

\begin{lstlisting}[caption=A class definition in JavaScript]
function Tile(letter) {
	this.letter = letter;
	this.saved = {};
}

Tile.prototype = {
	placeAt: function(x, y) {
		this.container.x = x;
		this.container.y = y;
		this.saved.x = x;
		this.saved.y = y;
	},

	moveTo: function(x, y, save) {
		// Moves the tile to the specified x and y coordinates.
		// This function is called by the Scramble and Builder classes
		createjs.Tween.get(this.container).to({x:x, y:y}, 500);
		if (save) {
			this.saved.x = x;
			this.saved.y = y;
		}
	},

	reset: function() {
		this.moveTo(this.saved.x, this.saved.y);
	},
};
\end{lstlisting}

\paragraph{Lua}
% AUTHOR (PARAGRAPH ON LUA): AFIYA
Lua does not directly implement object-oriented features, but these can be simulated using tables. Similar to objects, tables have a state and identity independent of their values. Two tables with the same value are two different objects. We can store different values within a table field and access it using the dot operator. Lua tables also have support for the \texttt{self} parameter which tells the method on which object it has to operate on. With the use of the self parameter, the same method can be used to act on many objects. Lua adds a hidden self parameter with the colon operator. There is no notion of a 'class' but prototype based inheritance can be implemented using metatables which are similar to JavaScript objects but are more powerful. 
