\section{Scripting Language Feature Comparison}
\label{comparison}

% Compare language features. Show examples for one or more language implementatinos
% whenever possible. The structure of this section will be a feature-by-feature comparison
% first, backed up by examples (or hypothetical examples) where possible.


\lettrine[nindent=0em,lines=3]{L}{orem} ipsum dolor sit amet, consectetur adipiscing elit. Curabitur eget bibendum lacus. Donec sem felis, suscipit a libero vitae, iaculis ultrices libero. Donec sollicitudin, mi vitae accumsan venenatis, magna sem pretium ligula, aliquet placerat lorem eros ac tortor. Suspendisse ultrices imperdiet fringilla. Pellentesque faucibus, turpis id gravida euismod, elit sem placerat mauris, quis lacinia eros augue ac mauris. Aliquam volutpat mauris sed orci tincidunt faucibus. Pellentesque ultricies vestibulum leo, a vehicula magna pretium nec. Sed feugiat massa id nunc aliquam condimentum. Duis mollis faucibus dolor.

\subsection{Lexical Structure}
% LEXICAL STRUCTURE
% STATEMENTS
% - Python: intentation sensitive; JavaScript: automatic semicolon insertion; 
%   Lua neither
% LUA: CHUNKS
% - Chunks: Lua specific thing
% COMMENTS

\subsection{Types, Values and Variables}
% TYPES, VALUES AND VARIABLES
\subsubsection{Numeric and Boolean Types}
% - number representation
%   - are there integer types or only double/float types?
%   - what is the behaviour of division? (eg: rounding, type of div by zero)

\subsubsection{Strings}
% - string representation
%   - regular strings vs raw strings (mention Python doc strings)
%   - "long strings"
% - single vs double quotes
% - Lua: length operator, JavaScript: length property

\subsubsection{Arrays, Lists, Tuples and Dictionaries}
% - Arrays, Lists, Tuples, Dictionaries
%   - only Python has all of these, Lua and JavaScript have tables/objects with 
%     special properties (covered under Objects)
%   - list comprehensions

\subsubsection{Objects and Tables}
% - object/table representation

\subsubsection{Special Types}
% Special Types
% - nil, null, undefined

\subsection{Expressions and Statements}
% EXPRESSIONS AND STATEMENTS
\subsubsection{Arithmetic, Relational and Logical Expressions}
% - operator overview (give a table showing operators for each language)
% - arithmetic, relational and logical expressions
%   - falsy, truthy types; in Lua 0 is true; object conversions during comparison
\subsubsection{String Concatenation}
% - string concatenation operators
% - Python join vs Lua concat
\subsubsection{Object and Array Initializers}
% - object and array initializers / creation
\subsubsection{Function Definition and Invocation}
% - function definition
% - invocation expressions
\subsubsection{Property Access}
% - property access
\subsubsection{Lua: Operator Overloading using Metatables}
% Lua operator oveloading
%   using Metatables

\subsection{Flow Control}
% FLOW CONTROL
\subsubsection{\texttt{if} Statements}
% - if statements
\subsubsection{\texttt{for} Statements}
% - for loops
% - "for each" loops (i.e. JavaScript for ... in ...; same with Lua)
\subsubsection{Assignment Statements}
% - multiple assignment (Python and Lua have it, JavaScript may get it)
\subsubsection{\texttt{break}, \texttt{return} and \texttt{goto} Statements}
% - break, return, goto/labelled loops

\subsection{Variable Scope}
% VARIABLE SCOPE
% - local variables vs global variables
% - block scope vs function scope
% - Lua do blocks

\subsection{Objects}
% OBJECTS
% - querying and setting properties
% - deleting properites
% - enumerating properties
% - property getters and setters
% - JavaScript and Lua arrays

\subsection{Functions}
% FUNCTIONS
% - how are arguments passed? (Python: must number of arguments match? Doesn't have to 
% in JavaScript and Lua)
% - Python's name-based arguments vs passing objects in JavaScript / Lua
% - scope
% - return value
% - functions as values
\subsubsection{Closures and Coroutines}
% - closures, coroutines

\subsection{Generators and Iterators}
% GENERATORS, ITERATORS

\subsection{Exception Handling}
% EXCEPTION HANDLING

\subsection{Object-Oriented Programming}
% OBJECT ORIENTED PROGRAMMING
% - class-based inheritance vs prototype based inheritance
% - private data

\subsection{Modules and Packages}
% MODULES AND PACKAGES