\section{Scripting Language Feature Comparison}
\label{comparison}

% Compare language features. Show examples for one or more language implementations
% whenever possible. The structure of this section will be a feature-by-feature comparison
% first, backed up by examples (or hypothetical examples) where possible.


\lettrine[nindent=0em,lines=3]{L}{orem} ipsum dolor sit amet, consectetur adipiscing elit. Curabitur eget bibendum lacus. Donec sem felis, suscipit a libero vitae, iaculis ultrices libero. Donec sollicitudin, mi vitae accumsan venenatis, magna sem pretium ligula, aliquet placerat lorem eros ac tortor. Suspendisse ultrices imperdiet fringilla. Pellentesque faucibus, turpis id gravida euismod, elit sem placerat mauris, quis lacinia eros augue ac mauris. Aliquam volutpat mauris sed orci tincidunt faucibus. Pellentesque ultricies vestibulum leo, a vehicula magna pretium nec. Sed feugiat massa id nunc aliquam condimentum. Duis mollis faucibus dolor.

\subsection{Lexical Structure}
% LEXICAL STRUCTURE
% STATEMENTS
% - Python: intentation sensitive; JavaScript: automatic semicolon insertion; 
%   Lua neither
% LUA: CHUNKS
% - Chunks: Lua specific thing
% COMMENTS

\subsubsection{Data Structures}
% - Arrays, Lists, Tuples, Dictionaries
%   - only Python has all of these, Lua and JavaScript have tables/objects with 
%     special properties (covered under Objects)
%   - list comprehensions

\paragraph{Python: Arrays, Lists, Tuples and Dictionaries}

\paragraph{JavaScript: Objects}

\paragraph{Lua: Tables}

\subsection{Variable Scope}
\label{varscope}
% VARIABLE SCOPE
% - local variables vs global variables
% - block scope vs function scope
% - Lua do blocks

\subsection{Functions}
% FUNCTIONS
% - how are arguments passed? (Python: must number of arguments match? Doesn't have to 
% in JavaScript and Lua)
% - Python's name-based arguments vs passing objects in JavaScript / Lua
% - scope
% - return value
% - functions as values
\subsubsection{Closures and Coroutines}
\label{closures}
% - closures, coroutines

\subsection{Generators and Iterators}
% GENERATORS, ITERATORS

\subsection{Object-Oriented Programming}
\label{oop}
% OBJECT ORIENTED PROGRAMMING
% - class-based inheritance vs prototype based inheritance
% - private data
% - JavaScript: no distinction between func and method: caling convention (same with Lua)
