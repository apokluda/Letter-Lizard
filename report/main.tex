%%% Preamble
\documentclass[letterpaper, 12pt, titlepage, final]{article}
\usepackage[english]{babel}             % English language/hyphenation
\usepackage{graphicx}

\graphicspath{{../mockups/}}
\graphicspath{{../screenshots/}}

\usepackage{amsmath}

\usepackage[sc]{mathpazo} % Use the Palatino font
\usepackage[T1]{fontenc} % Use 8-bit encoding that has 256 glyphs
\linespread{1.05} % Line spacing - Palatino needs more space between lines
\usepackage{microtype} % Slightly tweak font spacing for aesthetics

\usepackage[hmarginratio=1:1,top=32mm,columnsep=20pt]{geometry} % Document margins
%\usepackage{multicol} % Used for the two-column layout of the document
\usepackage[hidelinks]{hyperref} % For hyperlinks in the PDF

\usepackage[hang, small,labelfont=bf,up,textfont=it,up]{caption} % Custom captions under/above floats in tables or figures
\usepackage{subcaption}
\usepackage{booktabs} % Horizontal rules in tables
\renewcommand{\arraystretch}{1.2}

\usepackage{float} % Required for tables and figures in the multi-column environment - they need to be placed in specific locations with the [H] (e.g. \begin{table}[H])

\usepackage{lettrine} % The lettrine is the first enlarged letter at the beginning of the text
\usepackage{paralist} % Used for the compactitem environment which makes bullet points with less space between them

\usepackage{abstract} % Allows abstract customization
\renewcommand{\abstractnamefont}{\normalfont\bfseries} % Set the "Abstract" text to bold
\renewcommand{\abstracttextfont}{\normalfont\itshape} % Set the abstract itself to small italic text

\usepackage{titlesec} % Allows customization of titles
%\renewcommand\thesection{\Roman{section}}
%\titleformat{\section}[block]{\large\scshape\centering}{\thesection.}{1em}{} % Change the look of the section titles

\usepackage{fancyhdr} % Headers and footers
\pagestyle{fancy} % All pages have headers and footers
\fancyhead{} % Blank out the default header
\fancyfoot{} % Blank out the default footer
\fancyhead[C]{CS 798 - Scripting Languages Project Final Report} % Custom header text
\fancyfoot[C]{\thepage} % Custom footer text
\setlength{\headheight}{15pt}

% Setup TikZ

%\usepackage{tikz}
%\usetikzlibrary{arrows}
%\usetikzlibrary{topaths}
%\usetikzlibrary{positioning}
%\usetikzlibrary{scopes}
%\tikzstyle{block}=[draw opacity=0.7,line width=1.4cm]
%\newcount\mycount

\usepackage{placeins}
\usepackage{subcaption}
\usepackage{wasysym}
\usepackage{url}
\usepackage{rotating}

\usepackage{listings}
\renewcommand*{\lstlistlistingname}{List of Listings}
\usepackage{listings-lua}
\usepackage{color}
\definecolor{lightgray}{rgb}{0.95, 0.95, 0.95}
\definecolor{darkgray}{rgb}{0.4, 0.4, 0.4}
%\definecolor{purple}{rgb}{0.65, 0.12, 0.82}
\definecolor{editorGray}{rgb}{0.95, 0.95, 0.95}
\definecolor{editorOcher}{rgb}{1, 0.5, 0} % #FF7F00 -> rgb(239, 169, 0)
\definecolor{editorGreen}{rgb}{0, 0.5, 0} % #007C00 -> rgb(0, 124, 0)
\definecolor{orange}{rgb}{1,0.45,0.13}		
\definecolor{olive}{rgb}{0.17,0.59,0.20}
\definecolor{brown}{rgb}{0.69,0.31,0.31}
\definecolor{purple}{rgb}{0.38,0.18,0.81}
\definecolor{lightblue}{rgb}{0.1,0.57,0.7}
\definecolor{lightred}{rgb}{1,0.4,0.5}
\usepackage{upquote}
\usepackage{listings}
% CSS
\lstdefinelanguage{CSS}{
  keywords={color,background-image:,margin,padding,font,weight,display,position,top,left,right,bottom,list,style,border,size,white,space,min,width, transition:, transform:, transition-property, transition-duration, transition-timing-function},	
  sensitive=true,
  morecomment=[l]{//},
  morecomment=[s]{/*}{*/},
  morestring=[b]',
  morestring=[b]",
  alsoletter={:},
  alsodigit={-}
}

% JavaScript
\lstdefinelanguage{JavaScript}{
  morekeywords={typeof, new, true, false, catch, function, return, null, catch, switch, var, if, in, while, do, else, case, break},
  morecomment=[s]{/*}{*/},
  morecomment=[l]//,
  morestring=[b]",
  morestring=[b]'
}

\lstdefinelanguage{HTML5}{
  language=html,
  sensitive=true,	
  alsoletter={<>=-},	
  morecomment=[s]{<!-}{-->},
  tag=[s],
  otherkeywords={
  % General
  >,
  % Standard tags
	<!DOCTYPE,
  </html, <html, <head, <title, </title, <style, </style, <link, </head, <meta, />,
	% body
	</body, <body,
	% Divs
	</div, <div, </div>, 
	% Paragraphs
	</p, <p, </p>,
	% scripts
	</script, <script,
  % More tags...
  <canvas, /canvas>, <svg, <rect, <animateTransform, </rect>, </svg>, <video, <source, <iframe, </iframe>, </video>, <image, </image>, <header, </header, <article, </article
  },
  ndkeywords={
  % General
  =,
  % HTML attributes
  charset=, src=, id=, width=, height=, style=, type=, rel=, href=,
  % SVG attributes
  fill=, attributeName=, begin=, dur=, from=, to=, poster=, controls=, x=, y=, repeatCount=, xlink:href=,
  % properties
  margin:, padding:, background-image:, border:, top:, left:, position:, width:, height:, margin-top:, margin-bottom:, font-size:, line-height:,
	% CSS3 properties
  transform:, -moz-transform:, -webkit-transform:,
  animation:, -webkit-animation:,
  transition:,  transition-duration:, transition-property:, transition-timing-function:,
  }
}

\lstdefinestyle{htmlcssjs} {%
  % General design
%  backgroundcolor=\color{editorGray},
  basicstyle={\footnotesize\ttfamily},   
  frame=b,
  % line-numbers
  xleftmargin={0.75cm},
  numbers=left,
  stepnumber=1,
  firstnumber=1,
  numberfirstline=true,	
  % Code design
  identifierstyle=\color{black},
  keywordstyle=\color{blue}\bfseries,
  ndkeywordstyle=\color{editorGreen}\bfseries,
  stringstyle=\color{editorOcher}\ttfamily,
  commentstyle=\color{brown}\ttfamily,
  % Code
  language=HTML5,
  alsolanguage=JavaScript,
  alsodigit={.:;},	
  tabsize=2,
  showtabs=false,
  showspaces=false,
  showstringspaces=false,
  extendedchars=true,
  breaklines=true,
  % German umlauts
  literate=%
  {Ö}{{\"O}}1
  {Ä}{{\"A}}1
  {Ü}{{\"U}}1
  {ß}{{\ss}}1
  {ü}{{\"u}}1
  {ä}{{\"a}}1
  {ö}{{\"o}}1
}
%
\lstdefinestyle{py} {%
language=python,
literate=%
*{0}{{{\color{lightred}0}}}1
{1}{{{\color{lightred}1}}}1
{2}{{{\color{lightred}2}}}1
{3}{{{\color{lightred}3}}}1
{4}{{{\color{lightred}4}}}1
{5}{{{\color{lightred}5}}}1
{6}{{{\color{lightred}6}}}1
{7}{{{\color{lightred}7}}}1
{8}{{{\color{lightred}8}}}1
{9}{{{\color{lightred}9}}}1,
basicstyle=\footnotesize\ttfamily, % Standardschrift
numbers=left,               % Ort der Zeilennummern
%numberstyle=\tiny,          % Stil der Zeilennummern
%stepnumber=2,               % Abstand zwischen den Zeilennummern
numbersep=5pt,              % Abstand der Nummern zum Text
tabsize=4,                  % Groesse von Tabs
extendedchars=true,         %
breaklines=true,            % Zeilen werden Umgebrochen
keywordstyle=\color{blue}\bfseries,
frame=b,
commentstyle=\color{brown}\itshape,
stringstyle=\color{editorOcher}\ttfamily, % Farbe der String
showspaces=false,           % Leerzeichen anzeigen ?
showtabs=false,             % Tabs anzeigen ?
xleftmargin=17pt,
framexleftmargin=17pt,
framexrightmargin=5pt,
framexbottommargin=4pt,
%backgroundcolor=\color{lightgray},
showstringspaces=false,      % Leerzeichen in Strings anzeigen ?
}%
%

\usepackage{color}
\definecolor{mygreen}{rgb}{0,0.6,0}
\definecolor{mygray}{rgb}{0.5,0.5,0.5}
\definecolor{mymauve}{rgb}{0.58,0,0.82}

\lstset{ %
  backgroundcolor=\color{white},   % choose the background color; you must add \usepackage{color} or \usepackage{xcolor}
  basicstyle=\footnotesize,        % the size of the fonts that are used for the code
  breakatwhitespace=false,         % sets if automatic breaks should only happen at whitespace
  breaklines=true,                 % sets automatic line breaking
  captionpos=b,                    % sets the caption-position to bottom
  commentstyle=\color{mygreen},    % comment style
%  deletekeywords={...},            % if you want to delete keywords from the given language
  escapeinside={\%*}{*)},          % if you want to add LaTeX within your code
  extendedchars=true,              % lets you use non-ASCII characters; for 8-bits encodings only, does not work with UTF-8
  frame=single,                    % adds a frame around the code
  keepspaces=true,                 % keeps spaces in text, useful for keeping indentation of code (possibly needs columns=flexible)
  keywordstyle=\color{blue},       % keyword style
  language=JavaScript,                 % the language of the code
%  morekeywords={*,...},            % if you want to add more keywords to the set
  numbers=left,                    % where to put the line-numbers; possible values are (none, left, right)
  numbersep=5pt,                   % how far the line-numbers are from the code
  numberstyle=\footnotesize\color{mygray}, % the style that is used for the line-numbers
  rulecolor=\color{black},         % if not set, the frame-color may be changed on line-breaks within not-black text (e.g. comments (green here))
  showspaces=false,                % show spaces everywhere adding particular underscores; it overrides 'showstringspaces'
  showstringspaces=false,          % underline spaces within strings only
  showtabs=false,                  % show tabs within strings adding particular underscores
  stepnumber=1,                    % the step between two line-numbers. If it's 1, each line will be numbered
  stringstyle=\color{mymauve},     % string literal style
  tabsize=2,                       % sets default tabsize to 2 spaces
  title=\lstname                   % show the filename of files included with \lstinputlisting; also try caption instead of title
}

%%% Maketitle metadata
\newcommand{\horrule}[1]{\rule{\linewidth}{#1}}     % Horizontal rule

\title{
		%\vspace{-1in} 	
		\usefont{OT1}{bch}{b}{n}
		\normalfont \normalsize \textsc{University of Waterloo\\Cheriton School of Computer Science\\CS 798 - Scripting Languages Project Final Report} \\ [25pt]
		\horrule{0.5pt} \\[0.4cm]
		\huge A Comparison of Python, JavaScript and Lua Scripting Language Features \\
		\horrule{2pt} \\[0.5cm]
}
\author{
		\normalfont \normalsize
        Afiya Nusrat, Alexander Pokluda and Michael Wexler\\[-3pt] \normalsize
        March 31, 2014
}
\date{}

%%% Begin document
\begin{document}
\maketitle

\thispagestyle{fancy}
\pagenumbering{roman}

\begin{abstract}

In this report, we study three different scripting languages--Python, JavaScript and Lua--
and compare the similarities and differences of the features that they offer.
Our comparison is based on the implementation of a simple word game called Letter Lizard.
We have implemented Letter Lizard in each of the three languages using similar
data structures and algorithms as much as possible while taking advantage of each
language's idiomatic features where appropriate. We start with an overview of the
design and implementation of the game in each language and then compare the
features of each language using example code from each our Letter Lizard implementations.
We finish by noting which language features made certain aspects of the game
easier to implement and by summarizing the language features that we found most
useful. By comparing these three scripting languages through the implementation
of a non-trivial program, we hope to gain a deeper understanding of and appreciation
for scripting languages in general and the tasks for which they are best suited.

\end{abstract}

\cleardoublepage

% T A B L E   O F   C O N T E N T S
% ---------------------------------
\renewcommand\contentsname{Table of Contents}
\tableofcontents
\cleardoublepage
\phantomsection
%\newpage

% L I S T   O F   F I G U R E S
% -----------------------------
\addcontentsline{toc}{section}{List of Figures}
\listoffigures
\cleardoublepage
\phantomsection		% allows hyperref to link to the correct page
%\newpage

% L I S T   O F   L I S T I N G S
% -------------------------------
\addcontentsline{toc}{section}{List of Listings}
\lstlistoflistings
\cleardoublepage
\phantomsection		% allows hyperref to link to the correct page
%\newpage


\cleardoublepage

% set the page numbers to be arabic, starting at page 1 %
\setcounter{page}{1}
\pagenumbering{arabic}

%----------------------------------------------------------------------------------------
%	ARTICLE CONTENTS
%----------------------------------------------------------------------------------------

%\begin{multicols}{2} % Two-column layout throughout the main article text

\section{Introduction}
% Outline:
% - describe what Letter Lizard is, how it's played, etc
% - talk about the fact that three implementations of this game are used to compare the
%   language features
% - present screenshots from JavaScript implementation
% - lead into introduction of overview of three languages

\lettrine[nindent=0em,lines=3]{L}{orem} ipsum dolor sit amet, consectetur adipiscing elit. Curabitur eget bibendum lacus. Donec sem felis, suscipit a libero vitae, iaculis ultrices libero. Donec sollicitudin, mi vitae accumsan venenatis, magna sem pretium ligula, aliquet placerat lorem eros ac tortor. Suspendisse ultrices imperdiet fringilla. Pellentesque faucibus, turpis id gravida euismod, elit sem placerat mauris, quis lacinia eros augue ac mauris. Aliquam volutpat mauris sed orci tincidunt faucibus. Pellentesque ultricies vestibulum leo, a vehicula magna pretium nec. Sed feugiat massa id nunc aliquam condimentum. Duis mollis faucibus dolor.

\paragraph{Python}
%   - modern, doesn't include "unnecessary/redundant features" (such as += or ++), or
%     features that novice programmers are likely to abuse (such as goto)
%   - talk about history, popularity, insiprational languages, main features (eg
%     more traditional object oriented class-based design)
%   - talk about major projects using Python
%   - chosen for its popularity
%   - look at Wikipedia and language website for more information


\paragraph{JavaScript}
%   - "language of the Web"
%   - history, standartization process, insiprational languages, main features (eg
%     prototype-based inheritance), variable scoping (anti-feature), "everything
%     is an object"
%   - closures
%   - chosen for its ubiquitiy
%   - look at Wikipedia and language website for more information


\paragraph{Lua}
%   - shares many similaritise with javascript
%   - minimal but powerful
%   - local variable declared only where needed, "everything is a table"
%   - designed primarily as an extensible extension language
%   - chosen becasuse it is a fast, simple, versitile languages with many advanced 
%     features (eg. operator overloading)
%   - look at Wikipedia and language website for more information



\subsection{PyLetterLizard: Python Letter Lizard Implementation}

Python is a multipurpose, high-level programming language. Python has a friendly syntax, is easy-to-learn \cite{about_python}, and supports object-oriented, structured, and functional programming. It is used for a variety of applications \cite{whatis_python} by organizations such as Google, YouTube, NASA, and the New York Stock Exchange. The friendly nature of Python makes it a very good tool to be used in an educational setting, and many find it useful as a first programming language to learn.
		
	In the Python version of Letter Lizard (PyLetterLizard), we utilize three files, namely \texttt{letter\_lizard.py}, \texttt{game.py}, and \texttt{config.py}. These files all work in tandem to allow PyLetterLizard to operate correctly. To run the application, one can type \texttt{python letter\_lizard.py} at the command line. PyLetterLizard requires Python 2.7 to be installed, as well as the corresponding version of Pygame.  Upon launching, the function \texttt{main()} of \texttt{letter\_lizard.py} is called. Before this occurs, however, the files \texttt{config.py} and \texttt{game.py} are included. \texttt{config.py} declares configuration values that affect the placement of various game objects, as well as declares a couple of utility methods and objects. \texttt{game.py} declares a class Game which stores the state of a game in PyLetterLizard. We will discuss this module more in a moment.
	
	When \texttt{main()} of \texttt{letter\_lizard.py} executes, Pygame objects are instantiated, and some screen buttons are created. Then, the game enters an ``infinite'' loop, which does three things over and over: processes user input and events, updates the game state, and redraws the screen. We use the notion of a \emph{game state}, which is a value which reflects several states that the game can be in. This allows us to know when to draw objects that represent normal gameplay, as opposed to drawing the splash screen, menu screen, etc.
	
	As mentioned previously, \texttt{game.py} contains the class Game, which stores the game state, and contains several methods which allow the game state to be altered. The reason we decided to use a class to represent a game is because it allows for the creation of new games quite easily, and separates the functionality of a Game into a discrete structure.
	
	\texttt{game.py} contains several methods that \texttt{letter\_lizard.py} utilizes to update the game state and process input and events. The method \texttt{process\_letter} is called when a user types a letter, which checks to see if the letter exists in the set of letters, and updates the data structures correspondingly. \texttt{shuffle} is called when a user hits the space bar, which uses the \texttt{random.shuffle} function to randomly shuffle the puzzle word. This aids the in finding new words in the puzzle. The method \texttt{draw} allows an instance of the class Game to draw itself onto the screen. The screen object is passed into the draw method as an argument. By having the game object be in charge of drawing itself, we avoid having to include any drawing functionality in \texttt{letter\_lizard.py}, which gives the code tighter cohesion and looser coupling.
	
\subsubsection{Constructs of Python used}

%\noindent\begin{minipage}[t]{\textwidth}
%\centering
\begin{lstlisting}[language=Python, %
  caption=Basic constructs of Python used.]
    def process_backspace(self):
        self.message = ""
        if (len(self.letters_guessed) >= 1):
            letter_to_delete = self.letters_guessed[len(self.letters_guessed) - 1]
            del self.letters_guessed[len(self.letters_guessed) - 1]
            self.puzzle_letters_displayed[self.puzzle_letters_displayed.index('')] = letter_to_delete
\end{lstlisting}
%\end{minipage}
In the example above, we demonstrate several constructs of Python that we use in the Letter Lizard implementation. We demonstrate the ability for a Python class to define a member function (method), where the function has \texttt{self} as its argument. We use the \texttt{del} command in Python to delete the last element of the array \texttt{letters\_guessed}. This syntax for deleting array elements is a bit different than in other programming languages.

%\begin{minipage}[t]{1\linewidth}
%\centering
\begin{lstlisting}[language=Python, %
  caption=Demonstration of functional programming constructs in Python.]
    def __find_length_counts(self, words):
        word_lengths = [len(w) for w in words]
        return dict([(length, word_lengths.count(length)) for length in set(word_lengths)])
\end{lstlisting}
%\end{minipage}

In the above example, we have a private method \texttt{\_\_find\_length\_counts}, which when given an argument \texttt{words} (a list of words), will return a mapping from the unique counts of letters for each word to a count of how many words have that number of letters. For this method we utilize some functional aspects of Python. We use list comprehensions to iterate over all the lengths of \texttt{word\_lengths}. We transform \texttt{word\_lengths} into a set so that we only have the unique members. Then, we iterate over all the elements in that set, and for each one, create a tuple consisting of a mapping from the length of the word to the count of how many words have that length. The \texttt{dict()} construct transforms this list of tuples into a dictionary.
The above method is used for generating placeholders for the words to be found in a game.



\subsection{LetterLizardJS: JavaScript Letter Lizard Implementation}

% talk about how this version is implemented (eg classes, etc)
% show screenshots
% talk about implementation: language features that were useful,
% class diagrams, sequence diagrams, snippets of code, etc

\subsection{LuaLetterLizard: Lua Letter Lizard Implementation}

% talk about how this version is implemented (eg classes, etc)
% show screenshots
% talk about implementation: language features that were useful,
% class diagrams, sequence diagrams, snippets of code, etc

\section{Scripting Language Feature Comparison}
\label{comparison}

% Compare language features. Show examples for one or more language implementations
% whenever possible. The structure of this section will be a feature-by-feature comparison
% first, backed up by examples (or hypothetical examples) where possible.

% comparing the most major differences that we noticed using examples from our three implementations (when possible)
% don't have space to go into all the details
% First difference noted already: environments/frameworks led to different design

% NOT COVERED:
% Many differences: Types, Variables, Values (including Numeric and Boolean types, Strings)
% Special Types such as (nil, null and undefined)

% COVERED:
% Data Structures: arrays, lists, tuples, dictionaries (Python); Objects (JavaScript); Tables (Lua)

% NOT COVERED:
% Expressions and Statements, Flow Control

% COVERED: Variable Scope

% COVERED: Functions, closures, coroutines, generators, iterators

% NOT COVERED: Exception Handling, Modules and Packages

\lettrine[nindent=0em,lines=3]{N}{ow} that we have introduced the three implementations of
our Letter Lizard game, we will will compare and contrast the differences between
each language, noting their strengths and weaknesses, using example code from our
implementations where possible. When coding the Letter Lizard game in each language,
we tried to follow the same structure as much as possible, however, as was noted in section~\ref{lljs}
the event-driven environment of client-side JavaScript naturally led to a different
design for that implementation. 

\subsection{Lexical Structure}
% LEXICAL STRUCTURE
% STATEMENTS
% - Python: intentation sensitive; JavaScript: automatic semicolon insertion; 
%   Lua neither
% LUA: CHUNKS
% - Chunks: Lua specific thing
% COMMENTS

\subsubsection{Data Structures}
% - Arrays, Lists, Tuples, Dictionaries
%   - only Python has all of these, Lua and JavaScript have tables/objects with 
%     special properties (covered under Objects)
%   - list comprehensions

\paragraph{Python: Arrays, Lists, Tuples and Dictionaries}

\paragraph{JavaScript: Objects}

\paragraph{Lua: Tables}

\subsection{Variable Scope}
\label{varscope}
% VARIABLE SCOPE
% - local variables vs global variables
% - block scope vs function scope
% - Lua do blocks

\subsection{Functions}
% FUNCTIONS
% - how are arguments passed? (Python: must number of arguments match? Doesn't have to 
% in JavaScript and Lua)
% - Python's name-based arguments vs passing objects in JavaScript / Lua
% - scope
% - return value
% - functions as values
\subsubsection{Closures and Coroutines}
\label{closures}
% - closures, coroutines

\subsection{Generators and Iterators}
% GENERATORS, ITERATORS

\subsection{Object-Oriented Programming}
\label{oop}
% OBJECT ORIENTED PROGRAMMING
% - class-based inheritance vs prototype based inheritance
% - private data
% - JavaScript: no distinction between func and method: caling convention (same with Lua)


\section{Conclusion}
\label{conclusion}

\lettrine[nindent=0em,lines=3]{W}{e} have gained a deeper understanding and
appreciation for scripting languages by implementing a non-trivial program,
the Letter Lizard game, in three different scripting languages and comparing the
implementations. We found that it was fairly easy to implement the game in
all three languages and that the effort required was less that would have been
required to implement the game in static languages such as C, C++ and Java.
Of the three implementations, we found that JavaScript required the least
amount of effort due to the high-level functionality and event-driven nature 
of the client-side scripting environment provided by Web browsers. Of all the features
that we used, we found first-class functions and  closures to be the most useful. 
First-class functions and closures made writing callback functions for event-driven
programming simple and easy and led to a clean, modular design. We avoided some of
the negative features of JavaScript, such as automatic semicolon insertion, by
always finishing statements with an explicit semicolon. Although JavaScript objects
are not as powerful as Lua tables, we did not encounter any limitations in our
implementation. Similarly, although Python's ``broken lexical scoping'' may cause
problems for some programs, we did not encounter any difficulty with our implementation.
Overall, we gained a deeper understanding of and appreciation for scripting languages
and their ability to enable rapid application development and we will likely use
scripting languages more often in the future.

%----------------------------------------------------------------------------------------
%	REFERENCE LIST
%----------------------------------------------------------------------------------------

\nocite{*}
%\bibliography{report}
\bibliography{bibliography}
\bibliographystyle{plain}

%----------------------------------------------------

%\end{multicols}

%%% End document
\end{document}
