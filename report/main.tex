%%% Preamble
\documentclass[letterpaper, 12pt, titlepage, final]{article}
\usepackage[english]{babel}             % English language/hyphenation
\usepackage{graphicx}

\graphicspath{{../mockups/}}

\usepackage{amsmath}

\usepackage[sc]{mathpazo} % Use the Palatino font
\usepackage[T1]{fontenc} % Use 8-bit encoding that has 256 glyphs
\linespread{1.05} % Line spacing - Palatino needs more space between lines
\usepackage{microtype} % Slightly tweak font spacing for aesthetics

\usepackage[hmarginratio=1:1,top=32mm,columnsep=20pt]{geometry} % Document margins
%\usepackage{multicol} % Used for the two-column layout of the document
\usepackage[hidelinks]{hyperref} % For hyperlinks in the PDF

\usepackage[hang, small,labelfont=bf,up,textfont=it,up]{caption} % Custom captions under/above floats in tables or figures
\usepackage{subcaption}
\usepackage{booktabs} % Horizontal rules in tables
\renewcommand{\arraystretch}{1.2}

\usepackage{float} % Required for tables and figures in the multi-column environment - they need to be placed in specific locations with the [H] (e.g. \begin{table}[H])

\usepackage{lettrine} % The lettrine is the first enlarged letter at the beginning of the text
\usepackage{paralist} % Used for the compactitem environment which makes bullet points with less space between them

\usepackage{abstract} % Allows abstract customization
\renewcommand{\abstractnamefont}{\normalfont\bfseries} % Set the "Abstract" text to bold
\renewcommand{\abstracttextfont}{\normalfont\itshape} % Set the abstract itself to small italic text

\usepackage{titlesec} % Allows customization of titles
%\renewcommand\thesection{\Roman{section}}
%\titleformat{\section}[block]{\large\scshape\centering}{\thesection.}{1em}{} % Change the look of the section titles

\usepackage{fancyhdr} % Headers and footers
\pagestyle{fancy} % All pages have headers and footers
\fancyhead{} % Blank out the default header
\fancyfoot{} % Blank out the default footer
\fancyhead[C]{CS 798 - Scripting Languages Project Final Report} % Custom header text
\fancyfoot[C]{\thepage} % Custom footer text
\setlength{\headheight}{15pt}

% Setup TikZ

%\usepackage{tikz}
%\usetikzlibrary{arrows}
%\usetikzlibrary{topaths}
%\usetikzlibrary{positioning}
%\usetikzlibrary{scopes}
%\tikzstyle{block}=[draw opacity=0.7,line width=1.4cm]
%\newcount\mycount

\usepackage{placeins}
\usepackage{subcaption}
\usepackage{wasysym}

\usepackage{listings}
\renewcommand*{\lstlistlistingname}{List of Listings}
\usepackage{listings-lua}
\input{lststyle}

\lstset{ %
  backgroundcolor=\color{white},   % choose the background color; you must add \usepackage{color} or \usepackage{xcolor}
  basicstyle=\footnotesize,        % the size of the fonts that are used for the code
  breakatwhitespace=false,         % sets if automatic breaks should only happen at whitespace
  breaklines=true,                 % sets automatic line breaking
  captionpos=b,                    % sets the caption-position to bottom
  commentstyle=\color{mygreen},    % comment style
%  deletekeywords={...},            % if you want to delete keywords from the given language
  escapeinside={\%*}{*)},          % if you want to add LaTeX within your code
  extendedchars=true,              % lets you use non-ASCII characters; for 8-bits encodings only, does not work with UTF-8
  frame=single,                    % adds a frame around the code
  keepspaces=true,                 % keeps spaces in text, useful for keeping indentation of code (possibly needs columns=flexible)
  keywordstyle=\color{blue},       % keyword style
  language=JavaScript,                 % the language of the code
%  morekeywords={*,...},            % if you want to add more keywords to the set
  numbers=left,                    % where to put the line-numbers; possible values are (none, left, right)
  numbersep=5pt,                   % how far the line-numbers are from the code
  numberstyle=\tiny\color{mygray}, % the style that is used for the line-numbers
  rulecolor=\color{black},         % if not set, the frame-color may be changed on line-breaks within not-black text (e.g. comments (green here))
  showspaces=false,                % show spaces everywhere adding particular underscores; it overrides 'showstringspaces'
  showstringspaces=false,          % underline spaces within strings only
  showtabs=false,                  % show tabs within strings adding particular underscores
  stepnumber=2,                    % the step between two line-numbers. If it's 1, each line will be numbered
  stringstyle=\color{mymauve},     % string literal style
  tabsize=2,                       % sets default tabsize to 2 spaces
  title=\lstname                   % show the filename of files included with \lstinputlisting; also try caption instead of title
}

%%% Maketitle metadata
\newcommand{\horrule}[1]{\rule{\linewidth}{#1}}     % Horizontal rule

\title{
		%\vspace{-1in} 	
		\usefont{OT1}{bch}{b}{n}
		\normalfont \normalsize \textsc{University of Waterloo\\Cheriton School of Computer Science\\CS 798 - Scripting Languages Project Final Report} \\ [25pt]
		\horrule{0.5pt} \\[0.4cm]
		\huge A Comparison of Python, JavaScript and Lua Scripting Language Features \\
		\horrule{2pt} \\[0.5cm]
}
\author{
		\normalfont \normalsize
        Afiya Nusrat, Alexander Pokluda and Michael Wexler\\[-3pt] \normalsize
        March 31, 2014
}
\date{}


%%% Begin document
\begin{document}
\maketitle

\thispagestyle{fancy}
\pagenumbering{roman}

\begin{abstract}

In this report, we study three different scripting languages--Python, JavaScript and Lua--
and compare the similarities and differences of the features that they offer.
Our comparison is based on the implementation of a simple word game called Letter Lizard.
We have implemented Letter Lizard in each of the three languages using similar
data structures and algorithms as much as possible while taking advantage of each
language's idiomatic features where appropriate. We start with an overview of the
design and implementation of the game in each language and then compare the
features of each language using example code from each our Letter Lizard implementations.
We finish by noting which language features made certain aspects of the game
easier to implement and by summarizing the language features that we found most
useful. By comparing these three scripting languages through the implementation
of a non-trivial program, we hope to gain a deeper understanding of and appreciation
for scripting languages in general and the tasks for which they are best suited.

\end{abstract}

\cleardoublepage

% T A B L E   O F   C O N T E N T S
% ---------------------------------
\renewcommand\contentsname{Table of Contents}
\tableofcontents
\cleardoublepage
\phantomsection
%\newpage

% L I S T   O F   F I G U R E S
% -----------------------------
\addcontentsline{toc}{section}{List of Figures}
\listoffigures
\cleardoublepage
\phantomsection		% allows hyperref to link to the correct page
%\newpage

% L I S T   O F   L I S T I N G S
% -------------------------------
\addcontentsline{toc}{section}{List of Listings}
\lstlistoflistings
\cleardoublepage
\phantomsection		% allows hyperref to link to the correct page
%\newpage


\cleardoublepage

% set the page numbers to be arabic, starting at page 1 %
\setcounter{page}{1}
\pagenumbering{arabic}

%----------------------------------------------------------------------------------------
%	ARTICLE CONTENTS
%----------------------------------------------------------------------------------------

%\begin{multicols}{2} % Two-column layout throughout the main article text

\section{Introduction}
% Outline:
% - What is a scripting language? How do they compare to other languages?
%   - usually interpreted, or compiled on-the-fly to intermediate byte-code
%   - environments that can be automated through scripts: shells and Web browsers (perhaps the two best 
%     known examples), other general purpose examples exist such as Python; extension language: Lua
%   - "Lua is a language designed and widely used as an extension language. Python is a general-purpose 
%     language that is also commonly used as an extension language, while ECMAScript is still primarily a 
%     scripting language for web browsers, but is also used as a general-purpose language."
% - Why are they useful?
%   - "glue", cross-platform (web), rapid application development, extending application behaviour
% - Characteristics
%   - very high-level languages, high level of abstraction, dynamic
%   - dynamic: dynamically typed; interpreted; can execute code loaded over the network in the current
%     environment using an exec() method or something similar
%   - flexible type system; automatic memory managment

\lettrine[nindent=0em,lines=3]{S}{cripting} languages is an increasingly popular and important
category of programming languages. Programs written in scripting languages are often written for
a special run-time environment that can be automated through scripts--shells and Web browsers
are perhaps the two best known examples--or for a specialized domain, such as text processing.
JavaScript and Lua are examples of the former. JavaScript is used to extend the functionality
of Web pages displayed in a Web browser, while Lua is an extension language that is used in many 
commercial and free applications and is widely used in scripting video game engines. Lua is
often chosen for this task because it is designed to be very fast and easy to embed.
However, general-purpose scripting languages also exist and perhaps the most well-known one is Python.
Python is a widely-used, general-purpose high-level programming language that emphasizes code
readability and its syntax allows programmers to express concepts in fewer lines of code than would be 
possible in a language like C, enabling them to develop applications quicker. 
Scripting languages typically have a low barrier to entry and are easier for programers to 
get started with and provide a number of features that make them an important tool for increasing 
programmer productivity and distinguish them from programming languages such as C, C++ and Java. 
For example, scripting languages are generally very high-level languages that provide a high level of 
abstraction. They are usually dynamically typed and provide automatic memory management. Many
scripting languages can load and execute code dynamically in the context of the running program
by passing string consisting of program statements to an \texttt{exec()} function or something
similar.

% - basis for comparison is Letter Lizard:
%   - letter rearrangement game (show GUI mockup?)
%   - player is presented with a set of letters and the goal of the game is for the to form as
%     many words as possible from the set of letters before the timer runs out
%   - implemented in the three languages
%   - interface differs, but internal structure of the game is similar
%   - use game development framework for Python, EaselJS for JavaScript and text based UI for Lua
%   - can almost copy and paste much of our design document / proposal here

In this report, we study and compare three different scripting languages, Python,
JavaScript, and Lua, and compare the similarities and differences of the features that
they offer. Our comparison is based on the implementation of a simple letter rearrangement game
called Letter Lizard. In the game, the player is presented with a set of letters and 
their goal is to form as many dictionary words as possible from the set of letters
before the timer runs out. In order to make the game
more engaging and enjoyable, we implemented a number of features that enable the user to customize
their gameplay experience by providing options to set the number of rounds, the time per round, 
and the level of difficulty for each
game. Although wanted to create a game that is fun and enjoyable, our primary objective is to
compare and contrast the features of each programming language. Where possible, adhered to the same
design for each implementation while also using the idiomatic features of each language as much as 
possible. The game proceeds as follows:

\begin{itemize}
    \item On beginning the game, a welcome screen or a ``Splash Screen'' is displayed
    to the user. The user must hit the spacebar to proceed to the Main Menu screen.
    \item The Main Menu screen allows the user to configure game options, such as the
    number of rounds, time per round and level of difficulty. After setting the options, the
    user clicks \textbf{Start} to begin a new game and is taken to the main Game screen.
    \item The Game screen displays the set of letters to the user and allows them to type
    words that can be formed from the letters. It also shows the round number, the amount of
    time remaining, the user's score and placeholders for all of the dictionary words that
    can be formed from the set of letters. After typing a word, the user hits
    enter and the game engine checks to see that the word is a valid dictionary word, and, if so,
    reveals the word in the list of placeholders. The Game screen also has options to shuffle
    the set of letters and return to the main menu.
\end{itemize}

\begin{figure}
    \centering
    \begin{subfigure}{0.49\textwidth}
        \includegraphics[width=\textwidth]{Game_Screen.jpg}
        \caption{Main game screen mockup}
        \label{mainscreenmockup}
    \end{subfigure}
    \begin{subfigure}{0.49\textwidth}
        \includegraphics[width=\textwidth]{Gameplay.jpg}
        \caption{Gameplay mockup}
        \label{gameplaymockup}
    \end{subfigure}
    \caption{Two mockups from our design document demonstrating the proposed Letter Lizard game
    showing (a) the main game screen and (b) the gameplay.}
    \label{mockups}
\end{figure}

The Python and JavaScript implementations of Letter Lizard provide graphical user interfaces
that are modelled after the mockups presented in our design document. The mockups showing the
main game screen and gameplay are shown in Figure~\ref{mockups}. The Lua implementation provides
a text-based user interface\footnotemark,
but the user interaction parallels that of the other two versions.
Furthermore, the internal data structures and algorithms representing the game state and game play
are similar to the other two implementations.

\footnotetext{A ``screen'' in the text-based user interface is represented by several
lines of descriptive text printed to the console. The interface for the Lua implementation
of the game is discussed further in Section~\ref{luaimpl}.}

\paragraph{Python}
%   - modern, doesn't include "unnecessary/redundant features" (such as += or ++), or
%     features that novice programmers are likely to abuse (such as goto)
%   - talk about history, popularity, insiprational languages, main features (eg
%     more traditional object oriented class-based design)
%   - talk about major projects using Python
%   - chosen for its popularity
%   - look at Wikipedia and language website for more information


\paragraph{JavaScript}
%   - "language of the Web"
%   - history, standartization process, insiprational languages, main features (eg
%     prototype-based inheritance), variable scoping (anti-feature), "everything
%     is an object"
%   - closures
%   - chosen for its ubiquitiy
%   - look at Wikipedia and language website for more information


\paragraph{Lua}
%   - shares many similaritise with javascript
%   - minimal but powerful
%   - local variable declared only where needed, "everything is a table"
%   - designed primarily as an extensible extension language
%   - chosen becasuse it is a fast, simple, versitile languages with many advanced 
%     features (eg. operator overloading)
%   - look at Wikipedia and language website for more information



\subsection{PyLetterLizard: Python Letter Lizard Implementation}

Python is a multipurpose, high-level programming language. Python has a friendly syntax, is easy-to-learn \cite{about_python}, and supports object-oriented, structured, and functional programming. It is used for a variety of applications by organizations such as Google, YouTube, NASA, and the New York Stock Exchange~\cite{whatis_python}. The friendly nature of Python makes it a very good tool to be used in an educational setting, and many find it useful as a first programming language to learn.
		
	In the Python version of Letter Lizard (PyLetterLizard), we utilize three files, namely \texttt{letter\_lizard.py}, \texttt{game.py}, and \texttt{config.py}. These files all work in tandem to allow PyLetterLizard to operate correctly. To run the application, one can type \texttt{python letter\_lizard.py} at the command line in the PyLetterLizard directory in our Github repository (\texttt{uwaterloo-cs798scripting / group4}). PyLetterLizard requires Python 2.7 to be installed, as well as the corresponding version of Pygame.  Upon launching, the function \texttt{main()} of \texttt{letter\_lizard.py} is called. Before this occurs, however, the files \texttt{config.py} and \texttt{game.py} are included. \texttt{config.py} declares configuration values that affect the placement of various game objects, as well as declares a couple of utility methods and objects. \texttt{game.py} declares a class Game which stores the state of a game in PyLetterLizard. We will discuss this module more in a moment.
	
	When \texttt{main()} of \texttt{letter\_lizard.py} executes, Pygame objects are instantiated, and some screen buttons are created. Then, the game enters an ``infinite'' loop, which does three things over and over: processes user input and events, updates the game state, and redraws the screen. We use the notion of a \emph{game state} that is a value which reflects several states that the game can be in. This allows us to know when to draw objects that represent normal gameplay, as opposed to drawing the splash screen, menu screen, etc.
	
	As mentioned previously, \texttt{game.py} contains the class Game, which stores the game state, and contains several methods which allow the game state to be altered. The reason we decided to use a class to represent a game is because it allows for the creation of new games quite easily, and separates the functionality of a Game into a discrete structure.
	
	\texttt{game.py} contains several methods that \texttt{letter\_lizard.py} utilizes to update the game state and process input and events. When a player types a letter, the method \texttt{process\_letter} is called which checks to see if the letter exists in the set of letters displayed to the player, and updates the data structures correspondingly. \texttt{shuffle} is called when a user hits the space bar, which uses the \texttt{random.shuffle} function to randomly shuffle the puzzle word. This aids the player in finding new words in the puzzle. The method \texttt{draw} allows an instance of the class Game to draw itself onto the screen. The screen object is passed into the draw method as an argument. By having the game object be in charge of drawing itself, we avoid having to include any drawing functionality in \texttt{letter\_lizard.py}, which gives the code tighter cohesion and looser coupling.
	
\subsubsection{Constructs of Python used}
\label{pyconstructs}

%\noindent\begin{minipage}[t]{\textwidth}
%\centering
\begin{lstlisting}[language=Python, %
  caption=Basic constructs of Python used]
    def process_backspace(self):
        self.message = ""
        if (len(self.letters_guessed) >= 1):
            letter_to_delete = self.letters_guessed[len(self.letters_guessed) - 1]
            del self.letters_guessed[len(self.letters_guessed) - 1]
            self.puzzle_letters_displayed[self.puzzle_letters_displayed.index('')] = letter_to_delete
\end{lstlisting}
%\end{minipage}
In the example above, we demonstrate several constructs of Python that we use in the Letter Lizard implementation. We demonstrate the ability for a Python class to define a member function (method), where the function has \texttt{self} as its argument. We use the \texttt{del} command in Python to delete the last element of the array \texttt{letters\_guessed}. This syntax for deleting array elements is a bit different than in other programming languages.

%\begin{minipage}[t]{1\linewidth}
%\centering
\begin{lstlisting}[language=Python, %
  caption=Demonstration of functional programming constructs in Python]
    def __find_length_counts(self, words):
        word_lengths = [len(w) for w in words]
        return {length:word_lengths.count(length) for length in set(word_lengths)}
\end{lstlisting}
%\end{minipage}

In the above example, we have a private method \texttt{\_\_find\_length\_counts}, which when given an argument \texttt{words} (a list of words), will return a mapping from the unique counts of letters for each word to a count of how many words have that number of letters. For this method we utilize some functional aspects of Python. We use list comprehensions to iterate over all the lengths of \texttt{word\_lengths}. We transform \texttt{word\_lengths} into a set so that we only have the unique members. Then, we iterate over all the elements in that set, and for each one initialize an element of a new dictionary using a dictionary comprehension that maps a length value to the count of how many words have that length. The \texttt{dict()} construct transforms this list of tuples into a dictionary. The above method is used for generating placeholders for the words to be found in a game.



\FloatBarrier
\subsection{LetterLizardJS: JavaScript Letter Lizard Implementation}

% Introduce Language
% JAVASCRIPT
% - "language of the Web", implemented in Web browsers to allow client 
%   side scripts to interact with the user, control browser, preform 
%   asynchrous communication with server, alter docmuent that is displayed
% - also server-side programming, desktop and mobile applications
% - dynamic, prototype based
% - first class functions
% - syntax influenced by C, Java, but very different
% - key design principles takes from Self, Scheme
% - multi-paradigm: oop, imperative, functional
% - first introduced by Netscape, formalized as ECMAScript 

JavaScript is the ``language of the Web'' and has become an indispensable 
tool for Web developers. Client-side JavaScript scripts executed in Web
browsers are able to bring life to Web pages by interacting with the user,
controlling the behaviour of the browser, performing asynchronous communication
with Web servers and altering the document that is displayed. Although originally
introduced by Netscape for client-side scripting in 1995, it is increasingly
more common for JavaScript to be used in other contexts as well. For example,
a recent trend has been to implement server-side applications in JavaScript as
has been demonstrated by the explosion in popularity of Node.js~\cite{nodejs} and other
server-side JavaScript frameworks. Not long after Netscape started shipping 
JavaScript in its Navigator browser, it submitted the language to Ecma International
and it is now standardized as ECMA-262 and known as ECMAScript.
There are several well-known implementations of the language that conform to the standard.

% - supports structured programming like C
% - exeption: function scoping instead of block scoping
% - many think that this was not a good idea (cite Definitive guide?)
% - automatic semicolon insertion (also not a good idea) (cite def guide?)
% - dynamic typing: types associated vith values, not variables
% - object based: objections are associative arrays with String or int
%   property names (special langauge support for arrays)
% - aside from a few primitive types, everything is an object, including
%   funcions (which means that funcitons are first-class enitiies - they
%   can be assigned to variables and even returned from other functions
% - event driven; 1st class funcs used a lot in event-driven browser/
%   run-time environment
% - closures (give an example or refer to later section)
% - objects augmented with prototypes; can be used to simulate "classical"
%   oop features
% - no distinction between func and method: caling convention


JavaScript is a dynamic, prototype-based object oriented language with first-class
functions. Much of its syntax has been influenced by C, C++ and Java, but
the languages are actually very different. JavaScript borrows key design principles
from Self and Scheme. JavaScript supports several different programming
paradigms including object-oriented, imperative and functional. JavaScript supports
structured programming similar to C with many of the same flow-control statements
such as \texttt{if}, \texttt{for}, \texttt{while}, etc. One notable difference,
however, is the lack of block scoping for variables. Instead, JavaScript 
uses function scoping
for variables, which means that all variable declared in a function are
visible throughout the entire body of the function. This means that 
variables are even visible before they are declared. This feature 
is known as ``hoisting:'' JavaScript code behaves as if all variable 
declarations in a function are hoisted to the top of the function. This ``feature''
can easily cause bugs that are hard to find when local variables shadow variables in
an outer scope, and for this reason it is considered a negative aspect of the 
language~\cite{goodparts}. JavaScript also contains a mechanism that tries to
correct faulty programs by automatically inserting semicolons to complete statements, but quite
often this masks more serious errors~\cite{goodparts} or results in unexpected behaviour.
We explore these issues further in sections~\ref{varscope} and~\ref{exprandstmt} respectively.

JavaScript has a dynamic type system in which types are associated with values,
not variables. A few primitive types are provided by the language, such
as Number, String, Boolean, null and undefined. Aside from the primitive types,
everything else is an Object. Objects are composite types
that are comprised of properties: name-value pairs where the name is a String
(or an integer for arrays, we will see more about this in section~\ref{objects})
and the value is one of the primitive types or another object. Even functions
are objects (with associated behaviour), which means that functions are first-class
entities that may be assigned to variables and returned from other functions.

Each JavaScript function also contain a reference to the scope chain that was
in effect when the function was defined, which is used to resolve variable
names to values when the function is executed. A function, together with a
reference to its scope chain is know as a \emph{closure} (we will cover closures
in section~\ref{closures}). 
JavaScript usually runs in event-driven environments, such as the client-side
environment of a Web browser, that make heavy use of closures and first-class 
functions for callbacks.

JavaScript supports object-oriented programming, but not in the classical
sense. Rather than providing class-based inheritance, JavaScript provides
\emph{prototype-based inheritance}. Every object has as second object (or,
in some rare cases, \emph{null}) associated with it. This second object is
know as its \emph{prototype}, and the first object inherits properties from
its prototype. Many programmers who are new to JavaScript who come from a
C, C++, or Java background find JavaScript's prototype-based inheritance
confusing at first; however, it is really quite simple and works nicely
JavaScript's dynamic nature. Most classical object-oriented features can be
easily simulated in JavaScript. We discuss object-oriented programming
in section~\ref{oop}.

% talk about how this version is implemented (eg classes, etc)
% show screenshots
% talk about implementation: language features that were useful,
% class diagrams, sequence diagrams, snippets of code, etc

\begin{figure}
    \centering
    \begin{subfigure}{0.49\textwidth}
        \includegraphics[width=0.9\textwidth]{../screenshots/LetterLizardJS-SplashScreen2.png}
        \caption{Splash Screen}
        \label{lljssplash}
    \end{subfigure}
    \begin{subfigure}{0.49\textwidth}
        \includegraphics[width=0.9\textwidth]{../screenshots/LetterLizardJS-MainMenu2.png}
        \caption{Main Menu}
        \label{lljsmm}
    \end{subfigure}
    \begin{subfigure}{0.49\textwidth}
        \includegraphics[width=0.9\textwidth]{../screenshots/LetterLizardJS-Round1.png}
        \caption{Round Number message}
        \label{lljsround}
    \end{subfigure}
    \begin{subfigure}{0.49\textwidth}
        \includegraphics[width=0.9\textwidth]{../screenshots/LetterLizardJS-Gameplay2.png}
        \caption{Game Screen}
        \label{lljsgame}
    \end{subfigure}    
    \begin{subfigure}{0.49\textwidth}
        \includegraphics[width=0.9\textwidth]{../screenshots/LetterLizardJS-Hint2.png}
        \caption{Hint}
        \label{lljshint}
    \end{subfigure}
    \begin{subfigure}{0.49\textwidth}
        \includegraphics[width=0.9\textwidth]{../screenshots/LetterLizardJS-GoodJob3.png}
        \caption{Good Job message}
        \label{lljsgoodjob}
    \end{subfigure}
    \begin{subfigure}{0.49\textwidth}
        \includegraphics[width=0.9\textwidth]{../screenshots/LetterLizardJS-TimesUp.png}
        \caption{Time's Up message}
        \label{lljstimeup}
    \end{subfigure}
    \begin{subfigure}{0.49\textwidth}
        \includegraphics[width=0.9\textwidth]{../screenshots/LetterLizardJS-GameOver2.png}
        \caption{Game Over message}
        \label{lljsgameover}
    \end{subfigure}
    \caption{Screenshots from the Letter Lizard JavaScript implementation showing
    (a) the splash screen, (b) the main menu, (c) a round number message that is 
    displayed at the start of each round, (d) the main game screen as a user plays
    the game, (e) an in-game hint, (f) the ``Good Job!'' message that is displayed when
    a player finds all of the words, (g) the ``Time's Up!'' message that is displayed
    when the user does not find all of the words before the time runs out, and (h)
    the ``Game Over!'' message displayed at the end of the game.}
    \label{lljsscreenshots}
\end{figure}

Figure~\ref{lljsscreenshots} shows eight screenshots from LetterLizardJS. When the player
opens the Letter Lizard game in their browser, they are shown the Splash Screen
(Figure~\ref{lljssplash}) that provides information about the game and prompted to
press the spacebar to continue. After pressing the spacebar, the player is presented
with the Main Menu (Figure~\ref{lljsmm}) that allows them to set game options, namely
the number of rounds, time per round and level of difficulty. Once the game starts,
the player is presented with a set of letters represented as tiles on the game screen.
The player can move the tiles to form words by typing on the keyboard (Figure~\ref{lljsgame}).
At any time while playing the game, the player can shuffle the letters to help
her additional words by pressing the spacebar or clicking the \textbf{Shuffle} button
or request a hint by clicking the \textbf{Hint} button (Figure~\ref{lljshint}).
A number of in-game message are displayed to the user while the game is being played,
which are shown in Figures~\ref{lljsround} and \ref{lljsgoodjob}-\ref{lljsgameover}.

\begin{figure}
    \centering
	\includegraphics[scale=0.6]{../diagrams/LetterLizardJS-ClassDiagram.pdf}
	\caption{A class diagram showing the classes that make up the Letter
	Lizard JavaScript implementation. The event-driven, callback-based client-side scripting
	environment naturally led to a modularized, class-based design for this implementation.}
	\label{lljsclasses}
\end{figure}

\begin{figure}
    \centering
	\includegraphics[width=\textwidth]{../screenshots/LetterLizardJS-Legend.png}
	\caption{The on-screen representation of the Scramble, Builder, Tile and Word
	classes.}
	\label{lljslegend}
\end{figure}

One of the main differences that we first noticed between Python, Lua and 
JavaScript had to do with the event-driven nature of the client-side scripting
environment rather than the language itself: at the core of the the Python and
Lua implementations is a game loop that processes events, whereas the JavaScript
version does not have any such construct. Rather, the JavaScript version is
entirely event driven. The game loop in Python and Lua naturally led to a
procedural-based design, but the event-driven JavaScript environment
naturally led to a more modular, object-oriented design. Most of the functionality
of LetterLizardJS is implemented by the six classes shown in Figure~\ref{lljsclasses}.
The Tile, Scrabmle, Builder and Word classes also draw themselves on the main
game screen and their on-screen representation is shown in Figure~\ref{lljslegend}.
The Scramble class is responsible for creating Tiles for the letters that
will be shown to the user and initially places those tiles on itself on the
game screen. It also provides a shuffle method to rearrange the tiles when requested.
The Builder class moves the Tiles to form words when the user types on the keyboard
and also displays hints. The Game class is responsible for creating Word objects
to represent words to be found. It also gets the characters that the user has typed
from the Builder class and checks to see if they are valid words. If so, it shows
causes the corresponding Word object to show itself in the word list. The
Game class also updates the player's score when they find a word and manages
the game timer.

The code for our JavaScript implementation can be found in our Github repository
(uwaterloo-cs798scripting / group4) under the LetterLizardJS folder. To play the
game, simply need to load the \texttt{index.html} file in your browser; however,
due to the security policies of most Web browsers that restrict the functionality
of scripts loaded as local files, you will likely have to run an
HTTP server on your local machine and load the \texttt{index.html} file through
your server in order to play the game. Alternatively, we invite you to play the
game using a server that we have set up at using the following URL: 
\url{http://dahu.in/static/LetterLizardJS/index.html}.

% AT THE END: code can be found in github, we invite you to play at dahu.in...

\subsection{LuaLetterLizard: Lua Letter Lizard Implementation}
\label{luaimpl}

% Introduce Language
% LUA
%   - shares many similaritise with javascript
%   - minimal but powerful
%   - local variable declared only where needed, "everything is a table"
%   - designed primarily as an extensible extension language
%   - chosen becasuse it is a fast, simple, versitile languages with many advanced 
%     features (eg. operator overloading)
%   - look at Wikipedia and language website for more information

% talk about how this version is implemented (eg classes, etc)
% show screenshots
% talk about implementation: language features that were useful,
% class diagrams, sequence diagrams, snippets of code, etc

\section{Scripting Language Feature Comparison}
\label{comparison}

% Compare language features. Show examples for one or more language implementations
% whenever possible. The structure of this section will be a feature-by-feature comparison
% first, backed up by examples (or hypothetical examples) where possible.


\lettrine[nindent=0em,lines=3]{L}{orem} ipsum dolor sit amet, consectetur adipiscing elit. Curabitur eget bibendum lacus. Donec sem felis, suscipit a libero vitae, iaculis ultrices libero. Donec sollicitudin, mi vitae accumsan venenatis, magna sem pretium ligula, aliquet placerat lorem eros ac tortor. Suspendisse ultrices imperdiet fringilla. Pellentesque faucibus, turpis id gravida euismod, elit sem placerat mauris, quis lacinia eros augue ac mauris. Aliquam volutpat mauris sed orci tincidunt faucibus. Pellentesque ultricies vestibulum leo, a vehicula magna pretium nec. Sed feugiat massa id nunc aliquam condimentum. Duis mollis faucibus dolor.

\subsection{Lexical Structure}
% LEXICAL STRUCTURE
% STATEMENTS
% - Python: intentation sensitive; JavaScript: automatic semicolon insertion; 
%   Lua neither
% LUA: CHUNKS
% - Chunks: Lua specific thing
% COMMENTS

\subsubsection{Data Structures}
% - Arrays, Lists, Tuples, Dictionaries
%   - only Python has all of these, Lua and JavaScript have tables/objects with 
%     special properties (covered under Objects)
%   - list comprehensions

\paragraph{Python: Arrays, Lists, Tuples and Dictionaries}

\paragraph{JavaScript: Objects}

\paragraph{Lua: Tables}

\subsection{Variable Scope}
\label{varscope}
% VARIABLE SCOPE
% - local variables vs global variables
% - block scope vs function scope
% - Lua do blocks

\subsection{Functions}
% FUNCTIONS
% - how are arguments passed? (Python: must number of arguments match? Doesn't have to 
% in JavaScript and Lua)
% - Python's name-based arguments vs passing objects in JavaScript / Lua
% - scope
% - return value
% - functions as values
\subsubsection{Closures and Coroutines}
\label{closures}
% - closures, coroutines

\subsection{Generators and Iterators}
% GENERATORS, ITERATORS

\subsection{Object-Oriented Programming}
\label{oop}
% OBJECT ORIENTED PROGRAMMING
% - class-based inheritance vs prototype based inheritance
% - private data
% - JavaScript: no distinction between func and method: caling convention (same with Lua)


\section{Conclusion}

\lettrine[nindent=0em,lines=3]{L}{orem} ipsum dolor sit amet, consectetur adipiscing elit. Curabitur eget bibendum lacus. Donec sem felis, suscipit a libero vitae, iaculis ultrices libero. Donec sollicitudin, mi vitae accumsan venenatis, magna sem pretium ligula, aliquet placerat lorem eros ac tortor. Suspendisse ultrices imperdiet fringilla. Pellentesque faucibus, turpis id gravida euismod, elit sem placerat mauris, quis lacinia eros augue ac mauris. Aliquam volutpat mauris sed orci tincidunt faucibus. Pellentesque ultricies vestibulum leo, a vehicula magna pretium nec. Sed feugiat massa id nunc aliquam condimentum. Duis mollis faucibus dolor.


%----------------------------------------------------------------------------------------
%	REFERENCE LIST
%----------------------------------------------------------------------------------------

%\nocite{*}
%\bibliography{report}
%\bibliographystyle{plain}

%----------------------------------------------------

%\end{multicols}

%%% End document
\end{document}
