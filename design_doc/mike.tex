\section{Introduction}
    Cyberbullying is a pervasive problem in today's society, which has negative effects on users of technology, especially youth. Although there are many definitions in various sources, one can define as cyberbullying as a form of bullying or harassment which takes place over the Internet. This includes bullying over social networks, emails, text messaging, web forums, and mobile applications. Some examples of cyberbullying include: sending harassing text messages; spreading of rumours in a web forum; posting of embarrassing pictures without someone's consent; creation of fake profiles on social networks, with the intent to embarrass or humiliate someone \cite{what_is_cyberbullying}. 
    
    	Compared to traditional bullying, cyberbullying poses challenging issues that often worsen a situation.  One problem that makes cyberbullying particularly bad is that youth oftentimes don't equate joking around with bullying. Therefore, someone might send a seemingly innocuous message to another, thinking it is a joke, but the recipient of the message is actually offended and hurt by the message.  Cyberbullying is usually more impulsive than traditional bullying, and oftentimes youth do not realize the repercussions of their actions before the damage is already done \cite{what_is_cyberbullying}. Due to the speed of information transmission, and the virility of Internet messages, rumours or embarrassing photos can be spread very quickly. In the situation where an embarrassing rumour or photo is circulating around a social network, a youth could be embarrassed to go to school to face his or her peers \cite{cyber_vs_traditional}. This ultimately can lead to truancy, disciplinary problems, and depression for the victim.
    	
    	Patchin and Hinduja \cite{patchin2012} discuss some solutions that educators can make towards ameliorating the issue of cyberbullying. They mention that even though cyberbullying does not occur in the school, it can have a deteriorating effect on the morale of the students in the school, and can cause the school performance of the victim to suffer. They mention a couple different methods to foster a positive climate in a school setting, which would hopefully reduce the effect of cyberbullying:

\begin{itemize}
\item
Provide students with emotional support, a caring atmosphere, and emphasis on positive self-esteem
\item
Hold school assemblies regarding cyberbullying which students can heavily relate to. Emphasize that the vast majority of students use technology in responsible ways, while a small minority of students abuse technology and cyberbully others.
\item
Create a formal contract which students sign, where they promise not to cyberbully others
\item
Develop an anonymous method for students to report events where they are cyberbullied
\item
Create an anti-bullying awareness or pledge campaign
\end{itemize}

	The above suggestions ensure that students are educated about the harms of cyberbullying. One can argue that it is the schools' responsibility to educate students about proper use of the Internet. Although some may think that parents should ultimately be responsible for educating their children about Internet responsibility, there is no easy way to make sure that parents do so. Therefore, it is very important for schools to take the initiative to educate students regarding Internet responsibility and the harms of cyberbullying. It is in a school's best interest to reduce the amount of cyberbullying, in order to make sure that students emotional health is high, so that they can be better-educated.
    	
    The perceived anonymity of the web can often exacerbate the problem of cyberbullying \cite{anonymity}. Cyberbullies falsely believe that their actions are untraceable and that there will be no repercussions for their behaviour. This leads people to do and say things online that they would not do in a normal setting. \cite{taran} writes about an incident where a 12-year-old girl named Rebecca Ann Sedwick jumped from a platform at an abandoned cement factory to her death, after being tormented online and through mobile apps. Rebecca was constantly bombarded by text messages on her phone including, "Go kill yourself," and, "Why are you still alive?" The fact that these digital communications are anonymous made the bullies more likely to send these messages, not believing that there would be any repercussions. To provide some metrics on the extent of cyberbullying, a study found that about 10\% of teens engage in anonymous harassment \cite{collier}.
    
    The problem of anonymity on the Internet should not be underestimated when it comes to trying to understand the harms of cyberbullying. Several studies have shown that the probability that people behave badly increases when they are anonymous. The 1950's Milgram experiment, in which subjects were instructed to deliver supposedly painful shocks to students (actually actors), demonstrated that people have no problem following orders when instructed by someone of perceived authority. It also demonstrated that those who couldn't actually see their victims were more likely to deliver potentially lethal doses of electric shock. This demonstrates the power that perceived anonymity can have to cause a person to behave in a fashion that they would not in a different setting \cite{eldridge}. 
    
    	Several researchers suggest that the combination of relative anonymity of the Internet, combined with a lack of social cues present in a normal setting, is a major cause of cyberbullying on social media networks. 88\% of teens on social media sites have said that they have seen someone act cruelly towards another person on the social media site. Of those witnesses, 90\% said that they ignored it, and 35\% said that they ignore it frequently \cite{eldridge}. 
    	
	Some argue that there is a certain mob mentality to cyberbullying. 21\% of teens on social-media networks have admitted to joining in when they witnessed cruelty \cite{eldridge}. This mob mentality can often lead victims to suffer extensively for a long period of time. On Facebook, for instance, one can block a person who sends a harassing message. However, if there is a mob of cyberbullies, than blocking that one person will not prevent the others from bullying.

	Let's discuss some statistics of cyberbullying in order to get an idea of the scope of the problem. The Harford County Examiner reports that about half of teens have been victims of cyberbullying. This alarming statistic demonstrates how pervasive cyberbullying is amongst our youth. Also, fewer than 1 out of 5 cyber bullying incidents are reported to law enforcement \cite{stats}.  This demonstrates a fundamental issue in cyberbullying: it is extremely under-reported, and victims often don't take the action necessary in order to stop it. Victims often just ignore it, or brush it aside. It is the goal of this paper to offer some suggestions to allow for easier and more streamlined reporting of cyberbullying to law enforcement.
	
	As we have seen, cyberbullying is a challenging problem to solve, exacerbated by the anonymity that the Internet can provide. In order to ameliorate the problem of cyberbullying, there should be a better reporting system in place, which prevents and deters others from cyberbullying.