\section{Defence}
\subsection{Overview of defence}
\subsubsection{Centralized data}

Our design for this tracking system will ultimately form a centralized database which will store all incident report information in it. This database will thus help us in many aspects.

\subsubsection{Profiling}
Report records stored in the database can be inputed as a source to do data profiling. For example, we can have a better view on the social network distribution of cyberbullying through profiling the URLs those reports provide. We can also profile the age and education background of those reporters based on the information they filled in inside the form. Different kinds of studies and analysis can be conducted and processed by taking advantage of these first-handed data in our database. Thus, the features and characteristics of cyberbullying on social network can be further observed and studied. 

\subsubsection{Tracking}
Also, we can use this information for tracking purpose. Some people may be victims on several social networks or some people may be bullies and be reported several times. In our system, these records can be combined together to find out those who often bully on others. We can then have exact information about these people. We can obtain bullies' online information through their username and email address, and we can even track down their IP address with the help of local ISPs. This data, especially those snapshots that reporters uploaded can be extremely important evidence when needed. 

\subsubsection{Prevention}
Through this data, we can actually collect some useful information against those who often bully on others. Thus, certain pre-caution or prevention method can be applied upon them. We can even predict bullying incidents based on certain bullying patterns of those offenders. When the same email address or same IP we tracked surfaces several times on our system, we can actually inform the local ISP to prevent them from accessing the network or hand these cases to authorized organizations to deal with them. Some users in a website may be listed in the blacklist and be watched actively by web admins.

\subsection{Can be used by authorities like government and police department}
Our system can actually be a very useful third-party tool when a trusted and integrated party is in charge of this platform. So government or police department can benefit from this system and use it to further study or deal with the cyberbullying problem. Authorities can seek cooperation of local Internet Service Providers to track offenders’ IP address and their entire Internet activity histories when necessary. Legal action can be taken when severe cyberbullying incident happens. Once users know that government or police department is behind this system, it will give them more sense of safety to use this report system and will encourage more people to report. 

\subsection{Provide a supervision to web admins}
\subsubsection{Feedback system}
As the major actions e.g. to delete bullying content, to block the user or to ban the IP, will be taken by webserver admins, certain concerns regarding those admins may be raised. Whether these admins will finish their job in time or whether they will react properly to the bullies. Our system settled these problems down by introducing the feedback system, both for admins and for reporters. Our system needs constant feedback on those reports from the web admins to keep track of the process of solving these incidents.

\subsubsection{Pushing on admin’s work}
Once a report is filed, we will seek individual website admin to deal with that. They need to take immediate action to the reported incident. We can thus keep pushing them through the system asking them to update incident status from time to time to ensure that problem will be solved in time. Also, once this report is solved, we will need a feedback from the admin regarding this incident and notify the reporter. We will report that the incident is solved and also how it is solved. Then, we will let reporters to judge the work of web admins to see if they are satisfied with the result.  

\subsection{Progress feedback to encourage future report}
Most people who see cyberbullying happening on the social network or who are actually the victims of cyberbullying need to be encouraged to do the right thing. In order to encourage users to report cyberbullying incidents to the system, our system will keep them informed while the report is being processed. The result will be delivered to them with inspiring messages to let them know about their contribution to the whole Internet environment. If reporters are not satisfied with the result, they are always welcome to give feedback, and we will re-open the case to do further investigation. 

\subsection{Ensure identity anonymity to web admins}
Another benefit of this centralized platform is that the reporters don’t need to worry about their identity to be revealed to the web admin. After our system receive report form, we will only send information regarding the bullying incident and bully to the web admin. For example, we will only send bully’s username, email address and uploaded proofs to website admins without sending reporter’s personal information. In this case, website admin only gets the case with the necessary knowledge to solve the incident and is not able to know the reporter’s background information. In this way, reporters’ identity anonymity is achieved, which should give them more sense of safety when reporting cyberbullying.

\section{Examples}
In order to prove the feasibility of our system, we can give out a live example. Suppose Alice upload a video about herself on YouTube and someone, say Bob, a total stranger, just left some insulting comments to make fun of her under her video. She feels offended by those words and is afraid that her friends would see those comments. Then she goes to our website which is run by the local police department and fills out the report form to report Bob and uploads the screenshot with Bob’s comment in it. Once report form is received, an incident case is set up with status “processing” and Bob’s username and screenshot is sent back to YouTube’s admin. Admin gets that incident, he investigates into the incident to see if it did have happened and he also finds that user Bob has been reported several times with some severe insulting incidents. Besides deleting that comment, he also bans the associated IP address related to that username and reports back to our system. Then, based on the severity of this case, the police department decide to contact the local ISPs to find out Bob’s local address and take some legal action on him. Then the status of this incident case is updated as “closed” and feedback is sent to Alice describing how this case was dealt with. Finally, Alice is asked to send back the comment about whether she is satisfied with the result of the case. As comments are deleted, username Bob is banned from websites and the person behind this username may be under investigation. If she is satisfied with the result then this case is considered closed. If any problem is raised by Alice, the case will be reopened for further investigation. 

\section{Limitations and suggestions for future work}
\subsection{Response latency and system efficiency}
One concern of our system is that, when facing a high amount of reporting incidents, our system could probably be overloaded. Those web admins will suffer large amount of workload to deal with the incidents. They may not be able to handle all incidents in time as well as doing prevention and other daily routine work. Our central system needs to keep track and update the status of all incident to make sure those incidents are taken care of by right website admins. Hence, if the incoming incident reports piles up, it will directly affects both response time and solving time. Thus, the whole system efficiency will go down. It may take reporter quite a long time to get the feedback from the system. Also, whether certain action taken by website admins is proper or not, this still needs to be judged by our system. We still need a full evaluation system to evaluate the work done by the admins.  

\subsection{Combine external source with internal ones, collaborating with social media using same API of reporting system.}
As for now, social networks also have their own report system embedded in their web pages. People only need to click the report button beside the comments or person profiles to report that comment or person to web admins. These cases will then not go into our system but directly to web admins. There’s a way to combine both our system, an external reporting platform, and embedded internal report system together to make it a fully functional report environment. What we need to do is to collaborate with these social networks, inserting unified reporting APIs in their web pages. When user hit the report button inside those webpages, at the same time web admin receives that incident, the API will automatically send the report information to the our central database to store it as a record. The API will basically fill out the report form automatically, when reporting happens, with reporters’ information possibly matching our external report form.  

