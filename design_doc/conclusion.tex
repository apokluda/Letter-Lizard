	In this paper, we have discussed the issue of cyberbullying in our modern society. The advent of technology has brought the world many blessings. Increased productivity, connection to our loved ones, and educational advancements are just a couple of the cornucopia of good things that technology has brought us. However, in order for technology to help us in our lives, it must not be abused. Cyberbullying is one of several negative issues that technology has brought. Before the advent of technology, when a student was bullied in school, it ended after the student came home at 3pm. However, nowadays, children can be relentlessly bullied even while they are in their home, which is supposed to be a place where one should feel safe.
	
	We discussed our idea for a Cyberbullying Reporting Platform (CBRP). This idea is to allow for cyberbullying to be reported to the authorities as soon as it happens. This allows for prosecution of cyberbullies, which ultimately makes the Net a safer place for both children and adults. Increased prosecution of cyberbullies will ensure that those who commit these crimes are brought to justice, and acts as a deterrent for others who would consider bullying someone online. Our CBRP idea allows for greater communication between victims, police, and web admins. This gives victims a place to report online harassment, so that police and web admins can be notified, and proper action can be taken.
	
	As we mentioned, our idea has some limitations. One problem mentioned is that our system can be overloaded, due to the fact that there is a latency from the time when a user submits an incident, to the time that web admins can be notified and handle it. As a solution to this, we proposed the development of an API to our database, which allows an application to directly send incidents over HTTP. This would allow for every social network to allow their Report button to directly submit an incident to our database. This would allow for the instant logging of the alleged cyberbully's IP address, which would allow for easier tracking of cyberbullies.
	
	Overall, we believe our proposed system is a good launching point for a future cyberbullying tracking and reporting system to be used by the masses. Although it has some limitations, the concept at least provides an initial platform which can be modified over time to better serve the community. Eventually, we hope that a tool such as this can be developed to assuage the prevalent problem of cyberbullying.